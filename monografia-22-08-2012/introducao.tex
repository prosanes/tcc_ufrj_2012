Redes de Sensores Sem Fio (RSSFs) caracterizam-se pela formação de aglomerados de pequenos 
dispositivos que, atuando em conjunto, permitem monitorar ambientes físicos ou processos de 
produção com elevado grau de precisão. O desenvolvimento de aplicações que permitam explorar 
o uso dessas redes requer o estudo e a experimentação de protocolos, algoritmos e modelos de 
programação que se adequem às suas características e exigências particulares, entre elas, uso
de recursos limitados, adaptação dinâmica das aplicações, e a necessidade de integração com
outras redes, como a Internet.

Sistemas projetados para os dispositivos que formam as redes de sensores devem lidar apropriadamente
com as restrições e características particulares desses ambientes. 
A arquitetura adotada pelo TinyOS~\cite{tinyos/00} --- um dos sistemas operacionais mais usados
na pesquisa nessa área --- prioriza fortemente o tratamento dessas restrições em 
detrimento da simplicidade oferecida para o desenvolvimento de aplicações. 
A linguagem de programação usada é o nesC~\cite{nesc/03}, uma extensão de C que provê um modelo de programação baseado
em componentes e orietado a eventos.
%que provê baixo consumo de memória, optimizações, e previne condições de corrida.
%^^ Destacar depois em caracteristicas de nesC
Para lidar com as diversas operações de entrada e saída, o TinyOS utiliza um modelo de 
execução em duas fases, evitando bloqueios e, consequentemente, armazenamento de estados. 
A primeira fase da operação é um comando que pede ao hardware a execução de um serviço 
(ex.: sensoreamento). Este comando retorna imediatamente dando continuidade à execução. Quando o
serviço é terminado, o hardware envia uma interrupção, sinalizada como um evento pelo TinyOS. 
Então, o tratador do evento recebe as informações (ex.: valor sensoreado) e trata/processa essas informações conforme
programado\cite{LevisGay/09}. 
O problema gerado por essa abordagem é a falta da visão de um fluxo contínuo de execução 
na perspectiva do programador. 

O modelo de concorrência divide o código em dois tipos: assíncrono e síncrono. 
Um código assíncrono pode ser alcançável a partir de pelo menos
um tratador de interrupção. Em função disso, a execução desses trechos do programa pode ser interrompida
a qualquer momento e é necessário tratar possíveis condições de corrida.
Um código síncrono é alcançável somente a partir de tarefas (\textit{tasks}) que são 
procedimentos adiados (postergados). Uma tarefa executa até terminar (não existe concorrência entre elas), 
por isso as condições de corrida, neste contexto, são evitadas.  
As tarefas são todas escalonadas por um componente do TinyOS que usa uma política padrão de escalonamento 
do tipo \textit{First-in First-out}~\cite{LevisGay/09}.

Com o objetivo de oferecer maior flexibilidade aos desenvolvedores de aplicações, 
a versão mais atual do TinyOS (versão 2.1.x) trouxe novas facilidades.
Uma delas é a possibilidade de substituir o componente de escalonamento de tarefas
para implementar diferentes políticas de escalonamento~\cite{TEP106}.
A outra é a possibilidade de usar o modelo de programação multithreading,
 mais conhecido pelos desenvolvedores de aplicações e que pode
ser usado como alternativa para lidar com as dificuldades da programação orientada a eventos.

Neste trabalho avaliamos essas novas facilidades do TinyOS e propomos extensões que visam oferecer facilidade adicionais
para os desenvolvedores de aplicações.
Inicialmente, propusemos novos escalonadores de tarefas, implementando diferentes políticas de escalonamento por
prioridade. Avaliamos o modelo de multithreading oferecido, comparando diferentes formas de implementação
de uma aplicação básica e o custo da gerência de threads. Em seguida, tomando como base o modelo multithreading
oferecido, projetamos um mecanismo de gerência cooperativa de tarefas para o TinyOS baseado no conceito de co-rotinas.
Visamos uma solução alternativa entre o modelo de escalonamento de tarefas que executam até terminar, e o modelo de 
execução alternada entre as tarefas que permite maior flexibilidade durante a execução, mas com custo de gerência alto.

O modelo de gerência cooperativa de tarefas é uma solução apropriada para as redes de sensores sem fios devido à simplicidade
do hardware. Como os microcontroladores têm somente um núcleo, e não possuem tecnologia hyperthreading, não é possível 
existir duas unidades de execução executando em paralelo. Gerência cooperativa de tarefas permite manter contextos
distintos de execução e alternar entre eles de acordo com as necessidades da aplicação, minimizando as trocas de
contexto e eliminando a necessidade de mecanismos de sincronização.

Acreditamos que os escalonadores desenvolvidos oferecerão uma maior flexibilidade à programação, facilitando o
desenvolvimento de algoritmos complexos. Por meio de experimentos, constatamos que o custo de
escolamento aumentou de trinta a cem por cento, para uma quantidade razoável de vinte tarefas, dependendo do escalonador
utilizado.
A gerência cooperativa de tarefas desenvolvida facilitou a programação ao transformar os comandos de duas fases em
comandos bloqueantes de uma fase, e ao eliminar a necessidade de gerência do uso concorrente de recursos. 
Nos experimentos realizados, o tempo de processamento necessário para gerênciar rotinas cooperativas atingiu metade do
tempo necessário para gerências threads do modelo de programação multithreading.

Na seção \ref{teoria} abordamos conceitos básicos relacionados a redes de sensores sem fio, ao sistema operacional
TinyOS, e à sua linguagem de programação \textit{nesC}. Além disto, detalhamos as etapas da sequência de
inicialização, e descrevemos o modelo de concorrência.
Na seção \ref{escalonadorespropostos} fazemos uma introdução teórica sobre escalonamento de tarefas, e apresentamos os
escalonadores propostos. Por último, descrevemos os experimentos realizados e os resultados obtidos.
Na seção \ref{modelo-programacao} abordamos os conceitos de \textit{multithreading} e co-rotinas. Descrevemos o
uso e a implementação da biblioteca \textit{TOSThreads} que oferece um modelo de programação \textit{multithreding} ao
TinyOS. Depois, detalhamos a nossa implementação de gerência cooperativa de tarefas, e mostramos um exemplo de uso de
co-rotinas. Por último, descrevemos os experimentos realizados para comparar estes dois modelos e apresentamos os
resultados obtidos.
NA seção \ref{conclusoes} apresentamos algumas conclusões.

