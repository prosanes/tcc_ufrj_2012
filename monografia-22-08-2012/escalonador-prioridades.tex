Desenvolvemos um escalonador onde é possível estabelecer prioridades para as tarefas. 
A prioridade é passada como parâmetro através 
do comando \textit{postTask}. Quanto menor o número passado, maior a preferência da tarefa, sendo 0 a
mais prioritária e 254 a menos prioritária.
As \textit{Tasks} básicas também são aceitas, e são consideradas as tarefas de menor prioridade.

Foram encontrados dois problemas de \textit{starvation}. O primeiro relacionado com as tarefas básicas,
onde elas só seriam atendidas caso não houvesse nenhuma tarefa de prioridade na fila. Para resolver isso, foi definido um
limite máximo de tarefas prioritárias que podem ser atendidas em sequência. Caso esse limite seja excedido, uma tarefa
básica é atendida. O segundo é relacionado às próprias tarefas de prioridade. 
Se entrar constantemente \textit{tasks} de alta
prioridade, é possível que as de baixa prioridade não sejam atendidas. A solução se deu através do envelhecimento de
tarefas. Ou seja, \textit{tasks} que ficam muito tempo na fila, têm sua importância aumentada.

Dois tipos de estrutura de dados foram usadas para a organização das tarefas, uma fila comum e uma \textit{heap}. Com
isso, totalizou-se quatro diferentes versões do escalonador:
\begin{enumerate}
    \item Fila comum sem envelhecimento (anexo \ref{a:prioridades-fila})
    \item Fila comum com envelhecimento (anexo \ref{a:prioridades-fila-aging})
    \item Heap sem envelhecimento (anexo \ref{a:prioridades-heap})
    \item Heap com envelhecimento (anexo \ref{a:prioridades-heap-aging})
\end{enumerate}
A seguir uma tabela com a complexidade de inserção e remoção para cada escalonador:
\begin{center}
    \begin{tabular}{ | l | l | l | l | p{5cm} |}
    \hline
    Escalonador & Inserção & Remoção \\ \hline
    Fila, sem envelhecimento & $\bigcirc(n)$ & $\bigcirc(1)$ \\ \hline 
    Heap, sem envelhecimento & $\bigcirc(\log(n))$ & $\bigcirc(\log(n))$ \\ \hline
    Fila, com envelhecimento & $\bigcirc(n)$ & $\bigcirc(n)$ \\ \hline
    Heap, com envelhecimento & $\bigcirc(\log(n))$ & $\bigcirc(n)$ \\ \hline
    \end{tabular}
\end{center}
