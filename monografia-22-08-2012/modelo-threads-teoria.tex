\textit{TOSThreads} é uma biblioteca que permite programação com threads no TinyOS sem violar ou limitar o modelo de concorrência do
sistema. O TinyOS executa em uma única thread --- a thread do kernel --- enquanto a aplicação executa 
em uma ou mais threads --- nível de usuário.
Em termos de escalonamento, o kernel tem prioridade máxima, ou seja, a aplicação só executa quando o núcleo do sistema
está ocioso. Ele é responsável pelo escalonamento de todas as tarefas e execução das chamadas de sistemas. 
O escalonador de threads utiliza uma política \textit{Round-Robin} com um tempo padrão de 5 milisegundos. Ele oferece
toda a interface para manipulação de threads, como pausar, criar e destruir. 

Três tipos de fluxo de execução passam a existir: tarefas, interrupções e threads. Como foi visto na seção
\ref{modelo-concorrencia}, tarefas correspondem a um fluxo de execução, e tratadores de interrupção a outro. Além
disso, foi observado que os tratadores de interrupção podem interromper a execução de uma tarefa, porém o contrário não
é possível.
Com esta observação, pode-se dizer que tratadores de interrupção têm prioridade maior do que tarefas.
Para não violar o modelo de concorrência do TinyOS, as threads foram introduzidas com a menor prioridade de execução.
Isto significa que uma interrupção força a troca de contexto da thread, e caso seu tratador poste uma tarefa, esta
será executada antes da thread retomar o controle.

Trocas de Contextos
acontecem por três motivos diferentes: ocorrência de uma interrupção, termino do tempo de execução da thread, ou chamadas
bloqueantes ao sistema. 
Para implementar o primeiro caso, é inserida a função \textit{postAmble} ao final de todas as rotinas de processamento
de interrupção. Esta função verifica se foi postada uma nova tarefa, e caso positivo, o controle é passado para o
\textit{kernel}. Caso contrário, a thread continua a executar logo após o termino do tratador de interrupção.
Para implementar o segundo caso, é utilizado um temporizador que provoca uma interrupção ao final de cada
\textit{timeslice}. O tratador da interrupção posta uma tarefa, forçando o \textit{kernel} a assumir o controle e 
escalonar a próxima thread.

Chamadas de sistemas foram introduzidas para transformar chamadas de duas fases em chamadas de uma fase. Como os
serviços oferecidos pelo TinyOS são naturalmente \textit{split-phase}, estas chamadas devem ser bloqueantes.
Para fazer isto, a chamada de sistema bloqueia e adquire as informações da thread que a invocou. Posta uma tarefa que
executará o serviço \textit{split-phase} e acorda a thread do kernel. Eventualmente, a tarefa executará a primeira fase
do serviço. Na segunda fase, o resultado é enviado à thread, e esta é desbloqueada.

Para gerenciar o uso concorrente de recursos entre threads a seguintes primitivas são oferecidas:
\begin{description}
    \setlength{\itemsep}{0.3pt}
    \setlength{\parskip}{0pt}
      \setlength{\parsep}{0pt}
    \item[Mutex] Garante a exclusão mútua, como visto na seção \ref{multithread-corotinas}.
    \item[Semáforo] Garante uma ordem de execução, como visto na seção \ref{multithread-corotinas}.
    \item[Barreira] Garante que \textit{n} threads tenham chegado em um mesmo ponto. Todas threads que chamarem
        \textit{Barrier.block()} são bloqueadas até que \textit{n} chamadas tenham acontecido.
    \item[Variável de condição] Garante a suspensão de uma thread até que certa condição seja verdadeira.
    \item[Contador bloqueante] Garante a suspensão de uma thread até que o contador atinja o valor determinado.
\end{description}

O programador pode utilizar threads estáticas ou dinâmicas. A diferença está no momento de criação da pilha e do bloco de controle da
thread. Nas threads estáticas a criação é feita em tempo de compilação, enquanto nas threads dinâmicas 
a criação é feita em tempo de execução. O bloco de controle, também chamado de
\textit{Thread Control Block} (TCB), contém informações sobre a thread, como seu identificador, seu estado de
execução, o valor dos registradores (para troca de contexto), entre outras\cite{TEP134}.

