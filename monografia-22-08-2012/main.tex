\documentclass[a4paper, 10pt]{article}
\usepackage[utf8]{inputenc}
\usepackage[brazil]{babel}
\usepackage{ae}
\usepackage{verbatim}

\usepackage[pdftex]{graphicx} % Exposta para pdf e aceita figuras
%\usepackage{indentfirst} % Identa primeiro paragrafo
\usepackage{textcomp}
\usepackage[margin=30mm]{geometry}
\usepackage[pdfauthor={Pedro Rosanes},% Insere metadados com o nome do autor
	    pdftitle={Extensão dos mecanismos de gerência de tarefas do sistema operacional TinyOS},% Título que será mostrado na janela do PDF
	    pdftex]{hyperref} % Usa hiperlinks no decorrer do texto
%\parskip 7.2pt           % sets spacing between paragraphs
\renewcommand{\baselinestretch}{1.5}


\usepackage{listings}
\lstset{ %
language=C++,                % choose the language of the code
numbers=left, numberstyle=\tiny, stepnumber=2, numbersep=5pt,
basicstyle=\footnotesize,       % the size of the fonts that are used for the code
showspaces=false,               % show spaces adding particular underscores
showstringspaces=false,         % underline spaces within strings
showtabs=false,                 % show tabs within strings adding particular underscores
frame=single,                   % adds a frame around the code
tabsize=2,                  % sets default tabsize to 2 spaces
%captionpos=b,                   % sets the caption-position to bottom
%breaklines=true,                % sets automatic line breaking
%breakatwhitespace=false,        % sets if automatic breaks should only happen at whitespace
%title=\caption,                 % show the filename of files included with \lstinputlisting;
                                % also try caption instead of title
escapeinside={\%*}{*)},         % if you want to add a comment within your code
morekeywords={module,uses,interface,implementation,
                task,event,call,post,configuration,
                components,as,async,command,error_t,
                atomic,uint8_t,uint16_t,uint32_t,
                int8_t,int16_t,int32_t,
                signal,generic,provides},            % if you want to add more keywords to the set
belowskip=0pt
}

%---------------------------------------------------------------------------------------------------
\title{Extensão dos mecanismos de gerência de tarefas do sistema operacional TinyOS}
\author{Bolsista: Pedro Rosanes \and Orientador: Silvana Rossetto \and Departamento de Ciência da Computação}
\date{11 de abril de 2011}

\begin{document}

\begin{titlepage}
\maketitle
\tableofcontents
\end{titlepage}

\abstract{Resumo}
%Ta aparecendo resumo escrito nao sei pq!
Redes de Sensores Sem Fio (RSSFs) são formadas por pequenos dispositivos de sensoreamento, com 
espaço de memória e capacidade de processamento limitados, fonte de energia esgotável e comunicação sem fio.
O sistema operacional mais usado na programação desses dispositivos é o TinyOS, um sistema leve, 
projetado especialmente para consumir pouca energia, um dos requisitos mais importante para RSSFs. 
%O modelo  programação adotado pelo TinyOS é orientada a componentes (para facilitar a
%reutilização de código) e basedo em eventos (para economizar memória), 
O modelo de programação adotado pelo TinyOS prioriza o atendimento de interrupções.
Em função disso, as operações são normalmente divididas em duas fases: uma para envio
do comando, e outra para o tratamento da resposta (evento sinalizado via interrupção). 
Esse modelo de programação, baseado em eventos, quebra o fluxo de execução normal, dificultando a
tarefa dos desenvolvedores de aplicações. 
Para que os tratadores de eventos (interrupções) sejam curtos, tarefas maiores são
postergadas para execução futura e, para evitar concorrência entre elas,
as tarefas são executadas em sequência, uma após a outra (i.e., uma tarefa só é iniciada após a tarefa
anterior ser concluída).  
O objetivo deste trabalho é propor e implementar políticas alternativas de escalonamento de tarefas, e um modelo de
programação cooperativo para o TinyOS, visando a construção de abstrações de programação de nível mais alto que 
facilitem o desenvolvimento de aplicações nessa área.


%----------------------------------------------------------------------------------------------------
\section{Introdução}\label{intro}
Redes de Sensores Sem Fio (RSSFs) caracterizam-se pela formação de aglomerados de pequenos 
dispositivos que, atuando em conjunto, permitem monitorar ambientes físicos ou processos de 
produção com elevado grau de precisão. O desenvolvimento de aplicações que permitam explorar 
o uso dessas redes requer o estudo e a experimentação de protocolos, algoritmos e modelos de 
programação que se adequem às suas características e exigências particulares, entre elas, uso
de recursos limitados, adaptação dinâmica das aplicações, e a necessidade de integração com
outras redes, como a Internet.

Sistemas projetados para os dispositivos que formam as redes de sensores devem lidar apropriadamente
com as restrições e características particulares desses ambientes. 
A arquitetura adotada pelo TinyOS~\cite{tinyos/00} --- um dos sistemas operacionais mais usados
na pesquisa nessa área --- prioriza fortemente o tratamento dessas restrições em 
detrimento da simplicidade oferecida para o desenvolvimento de aplicações. 
A linguagem de programação usada é o nesC~\cite{nesc/03}, uma extensão de C que provê um modelo de programação baseado
em componentes e orietado a eventos.
%que provê baixo consumo de memória, optimizações, e previne condições de corrida.
%^^ Destacar depois em caracteristicas de nesC
Para lidar com as diversas operações de entrada e saída, o TinyOS utiliza um modelo de 
execução em duas fases, evitando bloqueios e, consequentemente, armazenamento de estados. 
A primeira fase da operação é um comando que pede ao hardware a execução de um serviço 
(ex.: sensoreamento). Este comando retorna imediatamente dando continuidade à execução. Quando o
serviço é terminado, o hardware envia uma interrupção, sinalizada como um evento pelo TinyOS. 
Então, o tratador do evento recebe as informações (ex.: valor sensoreado) e trata/processa essas informações conforme
programado\cite{LevisGay/09}. 
O problema gerado por essa abordagem é a falta da visão de um fluxo contínuo de execução 
na perspectiva do programador. 

O modelo de concorrência divide o código em dois tipos: assíncrono e síncrono. 
Um código assíncrono pode ser alcançável a partir de pelo menos
um tratador de interrupção. Em função disso, a execução desses trechos do programa pode ser interrompida
a qualquer momento e é necessário tratar possíveis condições de corrida.
Um código síncrono é alcançável somente a partir de tarefas (\textit{tasks}) que são 
procedimentos adiados (postergados). Uma tarefa executa até terminar (não existe concorrência entre elas), 
por isso as condições de corrida, neste contexto, são evitadas.  
As tarefas são todas escalonadas por um componente do TinyOS que usa uma política padrão de escalonamento 
do tipo \textit{First-in First-out}~\cite{LevisGay/09}.

Com o objetivo de oferecer maior flexibilidade aos desenvolvedores de aplicações, 
a versão mais atual do TinyOS (versão 2.1.x) trouxe novas facilidades.
Uma delas é a possibilidade de substituir o componente de escalonamento de tarefas
para implementar diferentes políticas de escalonamento~\cite{TEP106}.
A outra é a possibilidade de usar o modelo de programação multithreading,
 mais conhecido pelos desenvolvedores de aplicações e que pode
ser usado como alternativa para lidar com as dificuldades da programação orientada a eventos.

Neste trabalho avaliamos essas novas facilidades do TinyOS e propomos extensões que visam oferecer facilidade adicionais
para os desenvolvedores de aplicações.
Inicialmente, propusemos novos escalonadores de tarefas, implementando diferentes políticas de escalonamento por
prioridade. Avaliamos o modelo de multithreading oferecido, comparando diferentes formas de implementação
de uma aplicação básica e o custo da gerência de threads. Em seguida, tomando como base o modelo multithreading
oferecido, projetamos um mecanismo de gerência cooperativa de tarefas para o TinyOS baseado no conceito de co-rotinas.
Visamos uma solução alternativa entre o modelo de escalonamento de tarefas que executam até terminar, e o modelo de 
execução alternada entre as tarefas que permite maior flexibilidade durante a execução, mas com custo de gerência alto.

O modelo de gerência cooperativa de tarefas é uma solução apropriada para as redes de sensores sem fios devido à simplicidade
do hardware. Como os microcontroladores têm somente um núcleo, e não possuem tecnologia hyperthreading, não é possível 
existir duas unidades de execução executando em paralelo. Gerência cooperativa de tarefas permite manter contextos
distintos de execução e alternar entre eles de acordo com as necessidades da aplicação, minimizando as trocas de
contexto e eliminando a necessidade de mecanismos de sincronização.

Acreditamos que os escalonadores desenvolvidos oferecerão uma maior flexibilidade à programação, facilitando o
desenvolvimento de algoritmos complexos. Por meio de experimentos, constatamos que o custo de
escolamento aumentou de trinta a cem por cento, para uma quantidade razoável de vinte tarefas, dependendo do escalonador
utilizado.
A gerência cooperativa de tarefas desenvolvida facilitou a programação ao transformar os comandos de duas fases em
comandos bloqueantes de uma fase, e ao eliminar a necessidade de gerência do uso concorrente de recursos. 
Nos experimentos realizados, o tempo de processamento necessário para gerênciar rotinas cooperativas atingiu metade do
tempo necessário para gerências threads do modelo de programação multithreading.

Na seção \ref{teoria} abordamos conceitos básicos relacionados a redes de sensores sem fio, ao sistema operacional
TinyOS, e à sua linguagem de programação \textit{nesC}. Além disto, detalhamos as etapas da sequência de
inicialização, e descrevemos o modelo de concorrência.
Na seção \ref{escalonadorespropostos} fazemos uma introdução teórica sobre escalonamento de tarefas, e apresentamos os
escalonadores propostos. Por último, descrevemos os experimentos realizados e os resultados obtidos.
Na seção \ref{modelo-programacao} abordamos os conceitos de \textit{multithreading} e co-rotinas. Descrevemos o
uso e a implementação da biblioteca \textit{TOSThreads} que oferece um modelo de programação \textit{multithreding} ao
TinyOS. Depois, detalhamos a nossa implementação de gerência cooperativa de tarefas, e mostramos um exemplo de uso de
co-rotinas. Por último, descrevemos os experimentos realizados para comparar estes dois modelos e apresentamos os
resultados obtidos.
NA seção \ref{conclusoes} apresentamos algumas conclusões.



%----------------------------------------------------------------------------------------------------
\section{Conceitos Básicos}\label{teoria}

\subsection{Rede de Sensores Sem Fio}
Uma rede de sensores sem fio (RSSF) é um conjunto de dispositivos formando uma rede de comunicação \textit{ad-hoc}. Cada
sensor tem a capacidade de monitorar diversas propriedades físicas, como intensidade luminosa, temperatura, aceleração,
entre outras. Através de troca de mensagens, esses dispositivos podem agregar todas essas informações para detectar um evento
importante no local, como um incêndio. Essa conclusão é então encaminhada para um nó com maior capacidade computacional,
conhecido como estação base. Este nó pode decidir uma ação a ser tomada, ou enviar a informação pela Internet.

\begin{figure}
\centering
\includegraphics[scale=0.7]{images/sensores-e-topologia.png}
\caption{Rede de sensores sem fio}
\end{figure}

Estes sensores são desenvolvidos para monitoriar ambientes de difícil acesso, portanto, devem ser pequenos e utilizar
comunicação sem fio para facilitar a instalação no ambiente e minimizar o custo financeiro. Para evitar manutenções
frequêntes, eles também devem consumir pouca energia. Devido a estas características, o hardware destes dispositivos 
tende a ter recursos computacionais limitados. Ao invés de utilizar CPUs, são usados microcontroladores de 8 ou 16 bits, 
normalmente, com baixas frequências de relógio. Para armazenar o código da aplicação é utilizada uma pequena memória flash, 
da ordem de 100kB, e para as variáveis existe uma memória RAM, da ordem de 10kB. Os circuitos de rádio também têm uma 
capacidade reduzida de transferência, da ordem de kilobytes por segundos~\cite{LevisGay/09}.
Aliado ao hardware, o software também deve ser voltado para o baixo consumo de energia e de memória. Detalhes serão
vistos na próxima seção. 

Uma RSSF é usada para monitorar ambientes de difícil acesso, onde uma rede cabeada seria inviável ou custosa.
Alguns exemplos reais do uso de RSSF são: monitoramento da ponte Golden Gate em São Francisco, e dos vulcões Reventador e 
Tungurahua no Equador~\cite{LevisGay/09}. 



\subsection{TinyOS e nesC}
O TinyOS é o sistema operacional mais usado para auxiliar os programadores a desenvolverem aplicações para rede de
sensores sem fio de baixo consumo.
O modelo de programação provido é baseado em componentes e orientado a eventos.
Os componentes são pedaços de código reutilizáveis, onde são definidas claramente suas dependências e os serviços
oferecidos, por meio de interfaces. A linguagem \textit{nesc} implementa esse modelo estendendo a linguagem C.
É através da conexão (\textit{wiring}) de diversos componentes que o sistema é montado.

Já o modelo de programação orientado a eventos permite que o TinyOS rode uma aplicação, com somente uma linha de
execução, respondendo a diferentes interrupções de sistema, sem a necessidade de ações bloqueantes. 
Para isso todas as operações de entrada e saída são realizadas em duas fases.
Na primeira fase, o comando de E/S sinaliza para o hardware o que deve ser feito, e retorna imediatamente, dando continuidade a
execução. A conclusão da operação é sinalizada através de um evento, que será tratado pela segunda fase da operação de E/S.

O modelo de programação baseada em compontenes está intimamente ligada à programação orientada a eventos: Um componente
oferece a interface, implementando os comandos e sinalizando eventos relacionandos, enquanto outro componente
utiliza esta interface, através do uso dos comandos e da implementação dos tratadores de evento.

A divisão em componentes também facilita a implementação da camada de abstração de hardware. De forma que cada
plataforma tem um conjunto diferente de componentes para lidar com as instruções de cada hardware.
As abstrações providas são de serviços como sensoreamento, comunicação por rádio e armazenamento na memória flash.

\begin{figure}[htb]
\centering
\includegraphics[scale=0.4]{images/interfaces-componentes.png}
\caption{Ilustração de componentes e suas interfaces}
\end{figure}

\paragraph{Exemplo}
A aplicação \textit{Blink}, no anexo~\ref{a:Blink}, é utilizada para ilustrar estes conceitos básicos. Esta aplicação
faz os LEDs da plataforma piscarem continuamente.
Toda aplicação utiliza uma configuração para descrever os componentes que serão usados, e quais são as conexões entre
as interfaces.
\lstinputlisting[caption=Configuração (BlinkAppC.nc), label=l:BlinkAppC]{srcs/BlinkAppC.nc}

Na listagem \ref{l:BlinkAppC}, pode-se ver que alguns dos componentes utilizados são \textit{MainC}, \textit{BlinkC}, \textit{LedsC}.
\textit{MainC} é o responsável pela inicialização do sistema. E indica o termino deste
processo através do evento \textit{booted}, da interface \textit{Boot} (mais detalhes na seção \ref{inicializacao}).
\textit{BlinkC} implementa a lógica da aplicação.
E \textit{LedsC} implementa as operações necessárias para acender ou apagar os Leds da plataforma.

O outro componente utilizado é um temporizador. O comando \textit{new} é usado para
criar instâncias de componentes genéricos. Isso permite a criação de cópias distintas de uma mesma funcionalidade.
Neste caso são criados três temporizadores diferentes.
Alguns componentes não podem ter cópias distintas, como o \textit{LedsC} que representa uma estrutra física do
sensor, logo não pode ser multiplicada.

Na listagem \ref{l:BlinkAppC} também são definidas as conexões das interfaces de cada componente.
A construção da linha 14, por exemplo, indica que o módulo \textit{BlinkC} utiliza a interface
oferecida por \textit{LedsC} para apagar ou acender os LEDs. 

A lógica da aplicação está implementada no componente \textit{BlinkC}, através da contrução de um módulo..
\lstinputlisting[caption=Interfaces usadas pelo módulo (BlinkC.nc), label=l:BlinkC1, lastline=10]{srcs/BlinkC.nc}
Na listagem \ref{l:BlinkC1}, estão especificadas as interfaces utilizadas, que formaram as conexões vistas na
configuração.

\lstinputlisting[caption={Eventos, Comandos, Postagem de tarefas (BlinkC.nc)} 
                ,label=l:BlinkC2, firstline=11, lastline=25, firstnumber=11]{srcs/BlinkC.nc}
Finalmente, na listagem \ref{l:BlinkC2}, é definida a lógica do componente principal da aplicação.
O tratador do evento \textit{Boot.booted} é o primeiro a ser ativado após a inicialização do sistema.
Dentro deste, é invocado o comando para inicializar os temporizadores. A periodicidade de cada um é definida pelo
parâmetro passado. E por último, é postada uma tarefa, cujos detalhes serão vistos a seguir.
O tratador do evento \textit{Timer0.fired}, toda vez que é ativado, invoca o comando responsável por acender/apagar o
LED 0. O mesmo acontece para os outros temporizadores.

\paragraph{Tasks}
O TinyOS também utiliza o conceito de procedimento postergados, chamados de tarefas (\textit{tasks}). As próprias
tarefas, comando e tratadores de eventos podem postar uma nova tarefas, a qual é enfileirada para execução posterior.
Esta será atendida, de forma síncrona, pelo escalonador.
Ser executada de forma síncrona, significa que tarefas não são preemptiva entre si. Portanto, se diversas tarefas
compartilha uma mesma variável, não haverá condições de corrida. Mais detalhes serão vistos na seção
\ref{modeloconcorrencia}.

\lstinputlisting[caption=Implementação de tarefas (BlinkC.nc), label=l:BlinkC3, firstline=37, lastline=41, firstnumber=37]{srcs/BlinkC.nc}

Na listagem \ref{l:BlinkC2}, existe um exemplo de uma postagem de tarefa. E na listagem \ref{l:BlinkC3}, um exemplo da
implementação de uma tarefa, que neste caso somente emite uma mensagem de depuração.

\paragraph{Interfaces}
Ao desenvolver aplicações mais complexas, é preciso desenvolver componentes intermediários. Estes devem além de usar,
também devem oferecer interfaces. Para oferecer novas interfaces, o programador deve declará-las, implementar seus 
comandos, e sinalizar seus eventos. O componente que usar estas interfaces, será responsável por utilizar os comandos e
implementar os tratadores de eventos.
\begin{lstlisting}[caption=interface, label=interface]
interface Send {
    command error_t send(message_t* msg, uint8_t len);
    command error_t cancel(message_t* msg);
    event void sendDone(message_t* msg, error_t error);
    command uint8_t maxPayloadLength();
    command void* getPayload(message_t* msg, uint8_t len);
}

module SendExample {
    provides interface Send;
}
implementation {
    command error_t Send.send(message_t* msg, uint8_t len) {
        //Implementacao do comando send.
        ...

        signal Send.sendDone(msg_var, SUCCESS);
    }
    ...
}
\end{lstlisting}

Interfaces não são uma relação de um para um, e sim de \textit{n} para \textit{m}. Ou seja, diversos componentes podem
utilizar ou oferecer uma mesma interface. Isto permite que diversos códigos diferentes sejam chamados por somente um
comando. Esta propriedade é chamada de \textit{fan-out}. Por exemplo, suponha que dois componentes utilizam a interface
SplitControl (\ref{splitcontrol}). Logo, os dois componentes deveram implementar os tratadores dos eventos
\textit{startDone} e \textit{stopDone}. O \textit{fan-out} ocorre quando um evento é sinalizado, o que leva a execução dos 
dois tratadores. Esta propriedade também ocorre quando uma interface que é oferecida por mais de um componente, tem um
comando executado.

\begin{lstlisting}[caption=SplitControl, label=splitcontrol]
interface SplitControl {
    command error_t start();
    event void startDone(error_t error);
    
    command error_t stop();
    event void stopDone(error_t error);
}
\end{lstlisting}


\subsection {Sequência de inicialização do TinyOS}\label{inicializacao}
O principal componente do TinyOS, responsável por inicializar o sistema, é chamado \textit{MainC}. 
Ele inicializa os componentes de hardware e software e o escalonador de tarefas. Para isso, \textit{MainC} se liga aos
componentes \textit{RealmainP}, \textit{PlataformC}, \textit{TinySchedulerC}, e utiliza a interface \textit{SoftwareInit}.

\begin{figure}[htb]
\centering
\includegraphics[scale=0.4]{images/mainc.png}
\caption{MainC}
\end{figure}

Primeiro é configurado o sistema de memória e escolhido o modo de processamento. 
Com esses pré-requisitos básicos estabelecidos,  o escalonador de tarefas é inicializado 
para permitir que as próximas etapas possam postar tarefas.
O segundo passo é inicializar os demais componentes de hardware, permitindo a operabilidade da plataforma.
Alguns exemplos são configuração de pinos de entrada e saída, calibração do clock e dos LEDs.
Como esta etapa exige códigos específicos para cada tipo de plataforma, o MainC se liga ao componente
\textit{PlataformC} que implementa o tratamento requerido por cada tipo de plataforma.

\begin{figure}[htb]
\centering
\includegraphics[scale=0.45]{images/sequencia-de-inicializacao.png}
\caption{Sequência de Inicialização}
\end{figure}

O terceiro passo trata da inicialização dos componentes de software. 
Além de configurar os aplicativos básicos do sistema, como
os \textit{timer}s, nesta etapa são executados também os procedimentos de inicialização dos componentes 
da aplicação. Para isso, os componentes da aplicação que precisam ser inicializados devem oferecer a interface
{\em SoftwareInit}. Assim, durante a etapa de inicialização do sistema, os códigos de inicialização dos componentes da
aplicação são automaticamente chamados.
% !!! Esse paragrafo a cima parece ter muito a palavra inicialização !!!

Por último, quando todas as etapas foram concluídas, o MainC avisa a aplicação que a inicialização terminou, através do
evento \textit{Boot.booted()}. O TinyOS entra no
seu laço principal, no qual o escalonador espera por tarefas e as executa. É importante notar que
durante todo o processo de inicialização as interrupções do sistema ficam desabilitadas~\cite{TEP107}.

\begin{lstlisting}[caption=Código de inicialização]
module RealMainP {
    provides interface Booted;
    uses {
        interface Scheduler;
        interface Init as PlatformInit;
        interface Init as SoftwareInit;
    }
}
implementation {
     int main() __attribute__ ((C, spontaneous)) {
         atomic {
             platform_bootstrap();
             call Scheduler.init();
             call PlatformInit.init();
             while (call Scheduler.runNextTask());
             call SoftwareInit.init();
             while (call Scheduler.runNextTask());
         }
         __nesc_enable_interrupt();
         signal Boot.booted();
         call Scheduler.taskLoop();
         return -1;
     }
 \end{lstlisting}



\subsection{Modelo de concorrência do TinyOS}\label{modeloconcorrencia}
O TinyOS define o conceito de {\em tasks} (tarefas) como mecanismo central para
lidar com as questões de concorrência nas aplicações. 
Tarefas têm duas propriedades importantes. Elas não são preemptivas entre si, e são
executadas de forma adiada. Isso significa que ao postar uma tarefa, o fluxo de execução continua,
sem desvio, e ela só será processada mais tarde. 
Na definição básica do TinyOS, as tarefas não recebem parâmetros e não retornam resultados. 

O TinyOS minimiza os problemas clássicos de concorrência garantindo que qualquer possível condição de
corrida gerada a partir de tratadores de interrupção, seja detectada em tempo de compilação. 
Para que isso seja possível, o código em nesC é dividido em dois tipos:
\begin{description}
    \item[Código Assíncrono] Código alcançável a partir de pelo menos um tratador de interrupção.
    \item[Código Síncrono] Código alcançável somente a partir de tarefas.
\end{description}

Como visto na seção \ref{inicializacao}, ao final do processo de inicialização, as
interrupções são ativadas, e o evento \textit{Boot.booted} é sinalizado. O tratador deste evento, implementado no 
módulo principal da aplicação, é executado. E por último, o TinyOS entra em um laço infinito, onde as tarefas passam 
a ser atendidas. Este fluxo, quando não interrompido, é chamado de fluxo principal do TinyOS, e corresponde ao código
síncrono. Como não há preempção entre as tarefas, variáveis compartilhadas entre elas são imunes a condições de corrida.
Porém quando há uma interrupção de hardware, o tratador da interrupção assume o controle, e o fluxo principal é congelado 
até o termino daquele. Qualquer variável compartilhada, quando acessada por estes códigos assíncronos, está sujeita a condições
de corrida.

Como tarefas são postergadas, e atendidas pelo fluxo de execução principal, elas são usadas para fazer uma transição de
contexto assíncrono para síncrono.
Para fazer isto, um tratador de interrupção deve fazer somente o processamento mínimo, como transferência de dados entre o
\textit{buffer} e a memória. Após isto, deve postar uma tarefa para sinalizar o evento 
de conclusão da operação de E/S. Quando a tarefa for atentida, ela sinalizará o evento de forma síncrona. E portanto,
seu tratador também será um código síncrono.

Para auxiliar no controle de condições de corrida, qualquer código assíncrono devem ser marcados como \textit{async} no código fonte. 
Para contornar isto, deve-se usar o comando \textit{atomic} ou \textit{power locks}.

O comando \textit{atomic} garante exclusão mútua desabilitando interrupções. Dois fatos
importantes surgem com o seu uso, primeiro a ativação e desativação de interrupções consome ciclos de CPU. Segundo,
longos trechos atômicos podem atrasar outras interrupções, portanto é preciso tomar cuidado ao chamar outros componentes
a partir desses blocos.

Algumas vezes é preciso usar um determinado hardware por um longo tempo, sem compartilhá-lo. 
Como a necessidade de atomicidade não está no processador e sim no hardware, 
pode-se conceder sua exclusividade a somente um usuário (componente) através de
\textit{Power locks}. Para isso, primeiro é feito um pedido através de um comando, depois quando o recurso desejado
estiver disponível, um evento é sinalizado. Assim não há espera ocupada. 
Existe a possibilidade de requisição imediata. 
Nesse caso nenhum evento será sinalizado: se o recurso não estiver locado por outro usuário (componente), ele será
imediatamente cedido, caso contrário, o comando retornará falso. 
~\cite[Cap.11]{LevisGay/09}.




%----------------------------------------------------------------------------------------------------
\section{Escalonamento de tarefas}\label{escalonadorespropostos}
Nesta seção é feita uma abordagem teórica sobre escalonamento de tarefas.
Depois apresentamos a implementação do escalonador padrão de tarefas do TinyOS.
Também apresentamos o projeto e as etapas de implementação de novos escalonadores de tarefas para o TinyOS.
Implementamos três propostas: escalonador EDF (\textit{Earliest Deadline First}), escalonador com prioridades,  
e escalonador multi-nível.
Por último mostramos os experimentos e resultados obtidos, usados para comprar os diferentes escalonadores.

\subsection{Abordagem teórica sobre escalonamento de tarefas}\label{escalonador-teoria}
As tarefas, por serem procedimentos adiados, necessitam de algoritmos de escalonamento. Estes algoritmos também não
podem ser preemptivos, devido a natureza das tarefas do TinyOS.
O algoritmo mais simples, e também o padrão do TinyOS, é o \textit{First-Come, First-Served}, onde as tarefas são
atendidas segundo a ordem de chegada. O \textit{overhead} gerado é mínimo, e não há possibilidade de \textit{starvation}. 
Porém o tempo de resposta pode ser alto, se houver uma grande quantidade de tarefas na fila.

Escalonamento utilizando \textit{deadline} é muito usado em sistemas operacionais de tempo real~\cite{Stallings/04}. Neste algoritmo
a próxima tarefa a ser executada é aquela com menor prazo (\textit{deadline}). As diversas variações deste algoritmo utilizam
parâmetros como: tempo de entrada na fila de prontos, prazo para começar a tarefa, prazo para terminar a tarefa, tempo
de processamento, recursos utilizados, prioridade, existência de preempção.
Porém, o principal parâmetro utilizado pelos algoritmos é a existência ou não de preempção. Caso não exista preempção,
faz mais sentido utilizarmos, no escalonamento, o prazo para começar a tarefa. Caso exista preempção, o prazo para
terminar a tarefa é utilizado~\cite{Stallings/04}.
Um \textit{overhead} maior passa a existir, devido à ordenação das tarefas na fila e à preempção, caso exista.
Porém o tempo de resposta pode ser aproximadamente estipulado pela própria tarefa.

Em um escalonamento de prioridade fixa, cada tarefa indica, no momento de entrada para fila de prontos (tempo de
execução), sua importância em relação às
outras tarefas. Nestes algoritmos podemos ter preempção por parcela de tempo, na entrada de outras tarefas, ou não ter preempção. 
No primeiro tipo, pode existir um \textit{overhead} desnecessário quando o \textit{time-slice} da tarefa atual terminou,
porém não existe nenhuma outra com prioridade maior. O segundo tipo resolve este problema: se existe uma ordem de
tarefas na fila, esta ordem só pode ser alterada caso uma nova tarefa entre.
Quando não há preempção, a troca de tarefa só ocorre no término da execução de uma tarefa.
Neste escalonamento também há um \textit{overhead} maior, devido a ordenação das tarefas na fila.
A possibilidade de \textit{starvation} passa a existir, e o tempo de resposta varia de acordo com a prioridade das
tarefas.

O escalonamento de multi-nível é um caso especial do escalonamento de prioridade fixa. 
Cada tarefa determina seu nível de prioridade em tempo de compilação. Onde cada nível de prioridade tem uma fila, com 
política \textit{First-in First-out}, e as filas mais importantes devem ser atendidas por completo para que outra sejam
atendidas.

O escalonamento de prioridade dinâmica visa eliminar a possibilidade de \textit{starvation}. Neste caso, a tarefa ainda
indica sua importância no momento de entrada para fila de prontos. Porém, as tarefas que estão esperando para executar
aumentam de prioridade toda vez que não são atendidas. Apesar disto aumentar significativamente o \textit{overhead}, o
\textit{starvation} é eliminado.



\subsection{Escalonador padrão de tarefas do TinyOS}\label{escalonadorpadrao}
O componente responsável por gerenciar e escalonar tarefas no TinyOS é o componente {\em TinySchedulerC}.
O escalonador padrão adota uma política {\em First-in First-out} para agendar as tarefas. Ele também
cuida de parte do gerenciamento de energia, colocando a CPU em um estado de baixo consumo quando
não há nada para ser executado.

O programador, ao codificar uma tarefa, utiliza duas contruções:
\begin{lstlisting}[frame=none]
    post nome_da_tarefa();
    task void nome_da_tarefa() { //Definicao da tarefa }
\end{lstlisting}
Essas duas contruções são transformadas pelo compilador, fazendo com que a aplicação implemente uma interface chamada \textit{TaskBasic}.
A primeira é transformada em um comando, usado para indicar ao escalonador que esta tarefa deve entrar na fila.
O escalonador por sua vez, quando decidir que esta tarefa será a próxima a executar, sinalizará o evento relacionado a
este comando.
A segunda sintaxe é transformada no tratador deste evento, que implementa o que a tarefa deverá executar quando
escalonada.
É esta interface que permite a conexão das tarefas ao escalonador~\cite{LevisGay/09}.
\begin{lstlisting}
interface TaskBasic {
    async command error_t postTask();
    event void runTask();
}
\end{lstlisting}

Todo escalonador, além de prover a interface \textit{TaskBasic}, também deve prover a interface \textit{Scheduler}.
\begin{lstlisting}
interface Scheduler {
    command void init();
    command bool runNextTask();
    command void taskLoop();
}
\end{lstlisting}
A implementação dessas interfaces se dá da seguinte forma:
\begin{description}
    \item[command postTask()] Decide onde a tarefa será inserida na fila.
    \item[event runTask()] Indica que a tarefa deve executar.
    \item[command runNextTask()] Retira a primeira tarefa da fila e sinaliza sua execução com o evento
    \textit{runTask()}
    \item[command taskLoop()] Laço infinito que executa o comando \textit{runNextTask()}. Caso não haja tarefa para
    executar, coloca a CPU em modo de baixo consumo. 
\end{description}

Para criar novos tipos de tarefas, é preciso
definir uma interface nova, com o comando \textit{postTask} e o evento
\textit{runTask} que será provida pelo escalonador como visto acima. Por exemplo:
\begin{lstlisting}
interface TaskDeadline<precision_tag> { 
    async command error_t postTask(uint32_t deadline);
    event void runTask(); }
\end{lstlisting}

Na versão 2.1.x do TinyOS é possível mudar a política de gerenciamento de tarefas substituindo 
o componente escalonador padrão. Qualquer novo escalonador tem de
aceitar a interface de tarefa padrão (\textit{TaskBasic}), e garantir a execução de todas as tarefas 
(ausência de {\em starvation})~\cite{TEP106}.

O componente \textit{TinySchedulerC} é uma configuração que conecta as interfaces de tarefa à implementação do
escalonador.
Para alterar o escalonador basta definir um novo componente \textit{TinySchedulerC} e adicioná-lo ao diretório da
aplicação. Neste novo componente, a interface \textit{Scheduler} deve ser associada ao componente que implementa o
escalonador proposto, como ilutrado abaixo.

Por último, deve-se amarrar a interface da tarefa com a interface do escalonador. Por exemplo:
\begin{lstlisting}
configuration TinySchedulerC {
    provides interface Scheduler;
    provides interface TaskBasic[uint8_t id];
    provides interface TaskDeadline<TMilli>[uint8_t id];
}
implementation  {
    components SchedulerDeadlineP as Sched;
    ...
    Scheduler = Sched;
    TaskBasic = Sched; 
    TaskDeadline = Sched;
}
\end{lstlisting}

Um exemplo de aplicação que utiliza um escalonador novo pode ser visto no anexo \ref{a:appTeste}.
Para que o escalonador funcione corretamente no simulador TOSSIM é preciso adicionar funções que lidam com eventos no
simulador. Um exemplo desta extensão poder ser achada no arquivo
\textit{tinyOS-root-dir/tos/lib/tossim/SimSchedulerBasicP.nc}, ou no apêndice \ref{a:escalonador-tossim}


\subsection{Escalonador EDF (\textit{Earliest Deadline First})}\label{escalonadoredf}
Este escalonador (anexo \ref{a:edf})\footnote{O TEP 106~\cite{TEP106} disponibiliza um protótipo} aceita tarefas com deadline e 
elege aquelas com menor \textit{deadline} para executar. A interface usada para criar
esse tipo de tarefas é \textit{TaskDeadline}. O \textit{deadline} é passado por parâmetro pela função \textit{postTask}.
As tarefas básicas (\textit{TaskBasic}) também são aceitas, como recomendado pelo TEP 106\cite{TEP106}.

Em contraste, o escalonador não segue outra recomendação: não elimina a possibilidade de 
\textit{starvation} pois as tarefas
básicas só são atendidas quando não há nenhuma tarefa com \textit{deadline} esperando para executar. 
A fila é implementada da mesma forma que a do escalonador 
padrão\ref{escalonadorpadrao}, a única mudança está na inserção. Para
inserir, a fila é percorrida do começo até o fim, procurando-se o local exato de inserção.
Portanto, o custo de inserir é $\bigcirc(n)$, e o custo de retirar da fila é $\bigcirc(1)$. 
%Uma possível modificação seria utilizar uma \textit{heap}, mudando o custo de inserção  de 
%retirada para $\bigcirc(\logn)$.

%!!!explicar esse trecho melhor depois!!!
%A princípio tive problemas com o componente \textit{Counter32khzC}, 
%pois ele não existe para  a plataforma \textit{micaz}. Para poder compilar o
%escalonador foi preciso retirá-lo. Ele era usado para calcular a hora atual, e somar ao deadline. 
%Sem esse componente, temos um escalonador de prioridades (mínimo). 


\subsection{Escalonador por prioridades}\label{escalonadorprioridade}
Desenvolvemos um escalonador onde é possível estabelecer prioridades para as tarefas. 
A prioridade é passada como parâmetro através 
do comando \textit{postTask}. Quanto menor o número passado, maior a preferência da tarefa, sendo 0 a
mais prioritária e 254 a menos prioritária.
As \textit{Tasks} básicas também são aceitas, e são consideradas as tarefas de menor prioridade.

Foram encontrados dois problemas de \textit{starvation}. O primeiro relacionado com as tarefas básicas,
onde elas só seriam atendidas caso não houvesse nenhuma tarefa de prioridade na fila. Para resolver isso, foi definido um
limite máximo de tarefas prioritárias que podem ser atendidas em sequência. Caso esse limite seja excedido, uma tarefa
básica é atendida. O segundo é relacionado às próprias tarefas de prioridade. 
Se entrar constantemente \textit{tasks} de alta
prioridade, é possível que as de baixa prioridade não sejam atendidas. A solução se deu através do envelhecimento de
tarefas. Ou seja, \textit{tasks} que ficam muito tempo na fila, têm sua importância aumentada.

Dois tipos de estrutura de dados foram usadas para a organização das tarefas, uma fila comum e uma \textit{heap}. Com
isso, totalizou-se quatro diferentes versões do escalonador:
\begin{enumerate}
    \item Fila comum sem envelhecimento (anexo \ref{a:prioridades-fila})
    \item Fila comum com envelhecimento (anexo \ref{a:prioridades-fila-aging})
    \item Heap sem envelhecimento (anexo \ref{a:prioridades-heap})
    \item Heap com envelhecimento (anexo \ref{a:prioridades-heap-aging})
\end{enumerate}
A seguir uma tabela com a complexidade de inserção e remoção para cada escalonador:
\begin{center}
    \begin{tabular}{ | l | l | l | l | p{5cm} |}
    \hline
    Escalonador & Inserção & Remoção \\ \hline
    Fila, sem envelhecimento & $\bigcirc(n)$ & $\bigcirc(1)$ \\ \hline 
    Heap, sem envelhecimento & $\bigcirc(\log(n))$ & $\bigcirc(\log(n))$ \\ \hline
    Fila, com envelhecimento & $\bigcirc(n)$ & $\bigcirc(n)$ \\ \hline
    Heap, com envelhecimento & $\bigcirc(\log(n))$ & $\bigcirc(n)$ \\ \hline
    \end{tabular}
\end{center}


\subsection{Escalonador multi-nível}
No TinyOS, percebe-se uma divisão clara dos tipos de serviços: 
\begin{description}
    \item[Rádio] Comunicação sem fio entre diferentes nós da rede através de ondas de rádio.
    \item[Sensor] Sensoriamento de diferentes características do ambiente.
    \item[Serial] Comunicação por fio entre um nó e uma estação base (PC).
    \item[Básica] Outros serviços, como por exemplo temporizador.
\end{description}
Por isso, desenvolvemos um escalonador que divide as tarefas de acordo com os tipos definidos acima.
Onde cada tipo de tarefa utiliza uma interface distinta.
Cada tipo de tarefa tem sua própria fila com política \textit{First-in First-out}, e as filas mais importantes devem ser
atendidas por completo para que as outras sejam antendidas.
Uma aplicação de exemplo pode ser vista no anexo \ref{a:appTesteMulti}.

%!!!detalhar esse trecho melhor depois!!!
%Definiu-se que a ordem de prioridade seria serial, rádio, sensor e por último básica.




\subsection{Experimentos e resultados obtidos}
\paragraph{Experimentos com o escalonador de tarefas padrão}
Antes de começar a desenvolver outros escalonadores de tarefas, foi feito um experimento com o escalonador 
padrão que utiliza a política \textit{First in, First Out}.
Para medir a complexidade na prática, foi desenvolvida uma aplicação de teste (anexo~\ref{a:appTestePadrao}). Nela cada tarefa executa um loop de 65000
iterações, fazendo uma simples multiplicação em cada iteração. O número de tarefas variou entre 20, 50 e 100.
O tempo de execução foi medido em uma plataforma \textit{MicaZ}, utilizando o temporizador
\textit{Counter<TMicro,uint32\_t>}, utilizando uma precisão de microsegundos.
Os valores medidos não variaram mais de uma unidade entre diferentes execuções, e cada cenário foi executado dez
vezes.
\begin{center}
    \begin{tabular}{ | l | l | l | l | p{5cm} |}
    \hline
    Escalonador              & 20 Tarefas & 50 Tarefas & 100 Tarefas \\ \hline
    Escalonador Padrão       & 1366 & 1849 & 2652 \\ \hline
    \end{tabular}
\end{center}

%---------------------------------
\paragraph{Experimentos com o escalonador com prioridades}
Para avaliar o desempenho com o escalonador com prioridades foi desenvolvida a mesma aplicação de teste (anexo
\ref{a:appTeste}),
onde cada tarefa executa um loop de 65000 iterações, fazendo uma simples multiplicação em cada iteração. 
A prioridade de todas as tarefas, exceto uma, era igual, de forma que toda inserção deveria percorrer toda a fila.
A tarefa responsável por calcular o tempo de execução do experimento tinha a menor prioridade, para que esta fosse a
última a executar.
O número de tarefas variou entre 20, 50 e 100.
O tempo de execução foi medido em uma plataforma \textit{MicaZ}, utilizando o temporizador 
\textit{Counter<TMicro,uint32\_t>}, utilizando uma precisão de microsegundos. 
Os valores medidos não variaram mais de uma unidade entre diferentes execuções, e cada cenário foi executado dez
vezes.
\begin{center}
    \begin{tabular}{ | l | l | l | l | p{5cm} |}
    \hline
    Escalonador              & 20 Tarefas & 50 Tarefas & 100 Tarefas \\ \hline
    Escalonador Padrão       & 1366 & 1849 & 2652 \\ \hline 
    Fila, sem envelhecimento & 1733 & 4660 & 13721 \\ \hline 
    Heap, sem envelhecimento & 2603 & 4308 & 7486 \\ \hline
    Fila, com envelhecimento & 2278 & 7887 & 26066 \\ \hline
    Heap, com envelhecimento & 2665 & 4510 & 7887 \\ \hline
    \end{tabular}
\end{center}

Podemos perceber que, para um número pequeno de tarefas, a fila é mais eficiente que a heap.
%EXPLICAR
não é compensado o \textit{overhead} do algoritmo da heap.


%---------------------------------------------------------------------------------------------------
\section{Modelos de programação}\label{modelo-programacao}
Nesta seção, primeiro é feita uma abordagem teórica sobre Multithreading e co-rotinas. 
Também apresentamos a biblioteca \textit{TinyOS Threads}, que oferece um modelo de programação em threads, como
alternativa ao modelo orientado a eventos.
Depois apresentamos o projeto e as etapas de implementação de co-rotinas para o TinyOS.
Por último mostramos os experimentos e resultados obtidos, usados para comprar o modelo de threads com o de co-rotinas.

\subsection{Abordagem teórica sobre multithreading e co-rotinas}
\textit{Multithreading} refere-se a capacidade do sistema operacional e/ou do hardware de suportar diversas linhas de
execução, chamadas de \textit{threads}. Cada \textit{thread} contém um contexto que inclui instruções, variáveis, uma 
pilha de execução, e um bloco de controle. O suporte de diversar unidades de execução se dá por meio de paralelismo real
ou aparente. O primeiro tipo ocorre quando diferentes \textit{threads} executam em diferentes processadores, núcleos, ou
em processadores superescalares com multiplos bancos de registradores. O segundo tipo ocorre quando as \textit{threads}
intercalam o uso da CPU, por meio da gerência de um escalonador.

Em \textit{multithreading}, o escalonador faz uso de um artifício chamado preempção. Isso significa que uma thread em excução pode ser interrompida, 
após qualquer instrução, para ceder a CPU a outra \textit{thread}. Esta técnica permite que a CPU seja usada por todos, sem intervenção
do programador. Ou seja, a alternância de uso da CPU entre as \textit{threads} ocorre de forma independente ao código
implementado por elas.

Quando diferentes linhas de execução compartilham dados, o uso de preempção pode causar problemas de integridade destes
dados. Este problema, conhecido como condição de corrida, ocorre quando a preempção modifica a sequência de instruções
de uma operação. Para permitir que a operação execute sem interrupções, são utilizadas primitivas que desabilitam a
preempção temporariamente, garantindo a exclusão mútua de tais regiões.
Quando diversas \textit{threads} estão trabalhando em conjunto, as vezes é preciso garantir uma ordem de execução. Isso
é garantido com o uso de primitivas de sincronização.
Porém essas primitivas que gerenciam o uso concorrente de recurso são custosas.~\cite{Stallings/04}

Rotinas coopertativas, ou co-rotinas, têm as mesmas características das \textit{threads}, quando classificadas como
completas~\cite[s. 2.4]{Moura/04}. Porém elas cooperam no uso da CPU através de transferência explícita de controle. Com
isso elimina-se a necessidade de preempção, e consequentemente de gerêcia do uso concorrente de recursos.

Co-rotinas podem ser classificadas de acordo com o tipo de transferência de controle: simétricas e assimétricas. 
Co-rotinas do primeiro tipo têm a capacidade de ceder o controle para outra co-rotina explicitamente nomeada.
As assimétricas só podem ceder o controle para a co-rotina que lhes ativou e possuêm um comportamente semelhante ao
comportamento de funções.~\cite{Moura/04}

\label{multithread-corotinas}

\subsection{TinyOS Threads} \label{TOSThreads}
\textit{TOSThreads} é uma biblioteca que permite programação com threads no TinyOS sem violar ou limitar o modelo de concorrência do
sistema. O TinyOS executa em uma única thread --- a thread do kernel --- enquanto a aplicação executa 
em uma ou mais threads --- nível de usuário.
Em termos de escalonamento, o kernel tem prioridade máxima, ou seja, a aplicação só executa quando o núcleo do sistema
está ocioso. Ele é responsável pelo escalonamento de todas as tarefas e execução das chamadas de sistemas. 
O escalonador de threads utiliza uma política \textit{Round-Robin} com um tempo padrão de 5 milisegundos. Ele oferece
toda a interface para manipulação de threads, como pausar, criar e destruir. 

Três tipos de fluxo de execução passam a existir: tarefas, interrupções e threads. Como foi visto na seção
\ref{modelo-concorrencia}, tarefas correspondem a um fluxo de execução, e tratadores de interrupção a outro. Além
disso, foi observado que os tratadores de interrupção podem interromper a execução de uma tarefa, porém o contrário não
é possível.
Com esta observação, pode-se dizer que tratadores de interrupção têm prioridade maior do que tarefas.
Para não violar o modelo de concorrência do TinyOS, as threads foram introduzidas com a menor prioridade de execução.
Isto significa que uma interrupção força a troca de contexto da thread, e caso seu tratador poste uma tarefa, esta
será executada antes da thread retomar o controle.

Trocas de Contextos
acontecem por três motivos diferentes: ocorrência de uma interrupção, termino do tempo de execução da thread, ou chamadas
bloqueantes ao sistema. 
Para implementar o primeiro caso, é inserida a função \textit{postAmble} ao final de todas as rotinas de processamento
de interrupção. Esta função verifica se foi postada uma nova tarefa, e caso positivo, o controle é passado para o
\textit{kernel}. Caso contrário, a thread continua a executar logo após o termino do tratador de interrupção.
Para implementar o segundo caso, é utilizado um temporizador que provoca uma interrupção ao final de cada
\textit{timeslice}. O tratador da interrupção posta uma tarefa, forçando o \textit{kernel} a assumir o controle e 
escalonar a próxima thread.

Chamadas de sistemas foram introduzidas para transformar chamadas de duas fases em chamadas de uma fase. Como os
serviços oferecidos pelo TinyOS são naturalmente \textit{split-phase}, estas chamadas devem ser bloqueantes.
Para fazer isto, a chamada de sistema bloqueia e adquire as informações da thread que a invocou. Posta uma tarefa que
executará o serviço \textit{split-phase} e acorda a thread do kernel. Eventualmente, a tarefa executará a primeira fase
do serviço. Na segunda fase, o resultado é enviado à thread, e esta é desbloqueada.

Para gerenciar o uso concorrente de recursos entre threads a seguintes primitivas são oferecidas:
\begin{description}
    \setlength{\itemsep}{0.3pt}
    \setlength{\parskip}{0pt}
      \setlength{\parsep}{0pt}
    \item[Mutex] Garante a exclusão mútua, como visto na seção \ref{multithread-corotinas}.
    \item[Semáforo] Garante uma ordem de execução, como visto na seção \ref{multithread-corotinas}.
    \item[Barreira] Garante que \textit{n} threads tenham chegado em um mesmo ponto. Todas threads que chamarem
        \textit{Barrier.block()} são bloqueadas até que \textit{n} chamadas tenham acontecido.
    \item[Variável de condição] Garante a suspensão de uma thread até que certa condição seja verdadeira.
    \item[Contador bloqueante] Garante a suspensão de uma thread até que o contador atinja o valor determinado.
\end{description}

O programador pode utilizar threads estáticas ou dinâmicas. A diferença está no momento de criação da pilha e do bloco de controle da
thread. Nas threads estáticas a criação é feita em tempo de compilação, enquanto nas threads dinâmicas 
a criação é feita em tempo de execução. O bloco de controle, também chamado de
\textit{Thread Control Block} (TCB), contém informações sobre a thread, como seu identificador, seu estado de
execução, o valor dos registradores (para troca de contexto), entre outras\cite{TEP134}.



    \subsubsection{Exemplo de aplicação produtor/consumidor}
    Nesta seção ilustraremos o uso de threads no TinyOS, por meio de uma aplicação que utiliza o modelo produtor/consumidor.

Ao codificar o módulo principal, é preciso definir quantas threads serão utilizadas.
\lstinputlisting[firstline=4, lastline=10, firstnumber=4]{srcs/BenchmarkCThreads.nc}
E quais primitivas de gerência de concorrência.
\lstinputlisting[firstline=18, lastline=28, firstnumber=18]{srcs/BenchmarkCThreads.nc}

O próximo passo é inicializar estas primitivas, e as threads.
\lstinputlisting[firstline=33, lastline=45, firstnumber=33]{srcs/BenchmarkCThreads.nc}

A seguir a thread responsável por criar os produtos. Aqui podemos ver como é feito o uso das primitivas de gerência de
concorrência.
\lstinputlisting[firstline=72, lastline=90, firstnumber=72]{srcs/BenchmarkCThreads.nc}

A thread consumidora também é a responsável por acordar a thread que terminará de calcular o tempo de execução de todo o
programa.
\lstinputlisting[firstline=92, lastline=111, firstnumber=92]{srcs/BenchmarkCThreads.nc}

Ao final do consumo de todos os produtos a thread \textit{SerialSender} é desbloqueada.
Ela é responsável por calcular o tempo final de execução, e enviar este valor pela porta serial para um computador.
Na linha 60, podemos ver a utilização da chamada de sistema responsável por enviar este valor.
\lstinputlisting[firstline=47, lastline=69, firstnumber=47]{srcs/BenchmarkCThreads.nc}

Na configuração, é preciso declarar os componentes responsáveis pelas threads, além de definir o tamanho de suas pilhas.
\lstinputlisting[firstline=1, lastline=7]{srcs/BenchmarkAppCThreads.nc}
\lstinputlisting[firstline=21, lastline=23, firstnumber=21]{srcs/BenchmarkAppCThreads.nc}
O mesmo deve ser feito para as primitivas de gerência de concorrência.
\lstinputlisting[firstline=27, firstnumber=27]{srcs/BenchmarkAppCThreads.nc}



    \subsubsection{Implementação}
    A seguir, descrevemos detalhes da implementação da biblioteca \textit{TOSThread}. Mostraremos a organização dos diretórios e os
códigos fonte mais importantes.

\paragraph{Organização dos diretórios:}
O diretório raiz do \textit{TOSThread} é \textit{tinyos-root-dir/tos/lib/tosthreads/}.
Abaixo descrevemos sua estrutura básica de subdiretórios e as respectivas descrições\footnote{Todos os arquivos serão referenciados a partir do diretório
raiz \textit{tinyos-root-dir/tos/lib/tosthreads/}. i.e. \textit{types/thread.h}}:
\begin{description}
\setlength{\itemsep}{0.2pt}
\setlength{\parskip}{0pt}
\setlength{\parsep}{0pt}
    \item[chips:] Código específico de hardware.
    \item[interfaces:] Interfaces do sistema.
    \item[lib:] Extensões e subsistemas.
        \begin{description}
        \setlength{\itemsep}{0.2pt}
        \setlength{\parskip}{0pt}
        \setlength{\parsep}{0pt}
            \item[net:] Protocolos de rede (protocolos \textit{multihop}).
            \item[printf:] Componente que facilita a impressão de mensagens através da porta serial (para depuração).
            \item[serial:] Comunicação serial.
        \end{description}
    \item[platforms:] Código específico de plataformas.
    \item[sensorboards:] Drivers para placas de sensoreamento.
    \item[system:] Componentes do sistema.
    \item[types:] Tipos de dado do sistema (arquivos header).
\end{description}

\paragraph{Sequência de Boot:}
Na inicialização do \textit{TinyOS} com threads, primeiro há um encapsulamento da thread principal. Depois o curso
original é tomado.
A função \textit{main()} está implementada em \textit{system/RealMainImplP.nc}. A partir dela, o escalonador de threads
é chamado através de um signal.
\begin{lstlisting}
module RealMainImplP {
    provides interface Boot as ThreadSchedulerBoot;}
implementation {
    int main() @C() @spontaneous() {
        atomic signal ThreadSchedulerBoot.booted();}
}
\end{lstlisting}
O escalonador de threads, implementado em \textit{TinyThreadSchedulerP.nc} encapsula a atual linha de execução
como a thread do kernel. A partir de então, o curso normal de inicialização é executado. 
\begin{lstlisting}
event void ThreadSchedulerBoot.booted() {
    num_runnable_threads = 0;
    //Pega as informacoes da thread principal, seu ID.
    tos_thread = call ThreadInfo.get[TOSTHREAD_TOS_THREAD_ID]();
    tos_thread->id = TOSTHREAD_TOS_THREAD_ID;
    //Insere a thread principal na fila de threads prontas.
    call ThreadQueue.init(&ready_queue);

    current_thread = tos_thread;
    current_thread->state = TOSTHREAD_STATE_ACTIVE;
    current_thread->init_block = NULL;
    signal TinyOSBoot.booted();
}
\end{lstlisting}
Na fase final do \textit{boot}, é feita a inicialização do hardware, do escalonador de tarefas, dos componentes
específicos da plataforma, e de todos os componentes que se ligaram a \textit{SoftwareInit}. É então sinalizado que o 
\textit{boot} terminou, permitindo que o compontente do usuário execute. Por ultimo, o kernel passa o controle para o
escalonador de tarefas.
\begin{lstlisting}
void TinyOSBoot.booted() {
    atomic {
        //Inicializa hardware
        platform_bootstrap();
        call TaskScheduler.init();
        call PlatformInit.init();
        //Executa tarefas postas pela funcao a cima
        while (call TaskScheduler.runNextTask());
        call SoftwareInit.init();
        //Executa tarefas postas pela funcao a cima
        while (call TaskScheduler.runNextTask());
    }
    __nesc_enable_interrupt();
    //Sinaliza boot para o usuario
    signal Boot.booted();
    call TaskScheduler.taskLoop();
}
\end{lstlisting}
No escalonador de tarefas, quando não houver mais \textit{tasks} para executar, o controle é passado para o escalonador
de threads.
\begin{lstlisting}
command void TaskScheduler.taskLoop() {
    for (;;) {
        uint8_t nextTask;

        atomic {
            while((nextTask = popTask()) == NO_TASK) {
                call ThreadScheduler.suspendCurrentThread();
            }
        }
        signal TaskBasic.runTask[nextTask]();
    }
}
\end{lstlisting}

\paragraph{\textit{types/thread.h}:} 
Este arquivo contém os tipos de dados e constantes excenciais para threads. A seguir estão listados esses dados, e seus
respectivos códigos.
Estados que uma thread pode assumir, como ativo, inativo, pronto e suspenso.
\begin{lstlisting}
enum {
    TOSTHREAD_STATE_INACTIVE = 0,  //This thread is inactive and 
                                   //cannot be run until started
    TOSTHREAD_STATE_ACTIVE = 1,    //This thread is currently running 
                                   //on the cpu
    TOSTHREAD_STATE_READY = 2,     //This thread is not currently running, 
                                   //but is not blocked and has work to do 
    TOSTHREAD_STATE_SUSPENDED = 3, //This thread has been suspended by a 
                                   //system call (i.e. blocked)
};
\end{lstlisting}
Constantes que controlam a quantidade máxima de threads, e o periodo de preempção.
\label{thread_t}Estrutura da thread que contém dados como identificador, ponteiro para pilha, estado, ponteiro para função,
registradores.
\begin{lstlisting}
struct thread {
volatile struct thread* next_thread;  
    //Pointer to next thread for use in queues when blocked
thread_id_t id;                       
    //id of this thread for use by the thread scheduler
init_block_t* init_block;             
    //Pointer to an initialization block from which this thread was spawned
stack_ptr_t stack_ptr;                
    //Pointer to this threads stack
volatile uint8_t state;               
    //Current state the thread is in
volatile uint8_t mutex_count;         
    //A reference count of the number of mutexes held by this thread
uint8_t joinedOnMe[(TOSTHREAD_MAX_NUM_THREADS - 1) / 8 + 1]; 
    //Bitmask of threads waiting for me to finish
void (*start_ptr)(void*);             
    //Pointer to the start function of this thread
void* start_arg_ptr;                  
    //Pointer to the argument passed as a parameter to the start 
    //function of this thread
syscall_t* syscall;                   
    //Pointer to an instance of a system call
thread_regs_t regs;                   
    //Contents of the GPRs stored when doing a context switch
};
\end{lstlisting}
Estrutura para controle de chamadas de sistema. Contém seu identificador, qual thread está executando, 
ponteiro para função que a implementa.
\begin{lstlisting}
struct syscall {
struct syscall* next_call;        
    //Pointer to next system call for use in syscall queues when 
    //blocking on them
syscall_id_t id;                  
    //client id of this system call for the particular syscall_queue 
    //within which it is being held
thread_t* thread;                 
    //Pointer back to the thread with which this system call is associated
void (*syscall_ptr)(struct syscall*);   
    //Pointer to the the function that actually performs the system call
void* params;                     
    //Pointer to a set of parameters passed to the system call once it is 
    //running in task context
};
\end{lstlisting}

\paragraph{\textit{interfaces/Thread.nc}:} Contém os comandos de gerenciamento da thread e um evento para executá-la.
Estes comandos permitem começar, terminar, pausar ou resumir a execução da thread.
\begin{lstlisting}
interface Thread {
    command error_t start(void* arg);
    command error_t stop();
    command error_t pause();
    command error_t resume();
    command error_t sleep(uint32_t milli);
    event void run(void* arg);
    command error_t join();
}  
\end{lstlisting}

\paragraph{\textit{interfaces/ThreadInfo.nc}:} Contém comandos para receber ou apagar as informações da thread,
vistas em \ref{thread_t}. 
\begin{lstlisting}
interface ThreadInfo {
    async command error_t reset();
    async command thread_t* get();
} 
\end{lstlisting}

\paragraph{\textit{interfaces/ThreadScheduler.nc}:} Contém os comandos para gerenciar todas as threads. Essas funções
servem para obter informações das threads, inicializá-las e trocar de contexto.
Alguns comandos de \textit{interfaces/Thread.nc} são simplesmente mapeados para os comandos abaixo.
\begin{lstlisting}
interface ThreadScheduler {
    //Comandos para obter informacoes de uma thread
    async command uint8_t currentThreadId();
    async command thread_t* currentThreadInfo();
    async command thread_t* threadInfo(thread_id_t id);

    //Comandos para gerenciar a execucao de uma thread
    //Estes sao usados pelas proprias threads
    command error_t initThread(thread_id_t id);
    command error_t startThread(thread_id_t id);
    command error_t stopThread(thread_id_t id);

    //Comandos para gerenciar a execucao de uma thread
    //Estes sao usados por tratadores de interrupcao ou syscalls
    async command error_t suspendCurrentThread();
    async command error_t interruptCurrentThread();
    async command error_t wakeupThread(thread_id_t id);
    async command error_t joinThread(thread_id_t id);
}
\end{lstlisting}

\paragraph{\textit{system/ThreadInfoP.nc}:}\label{ThreadInfoP} Contém o vetor que representa a pilha, as informações da thread,
como visto em \ref{thread_t} e a função que sinaliza a execução.
\begin{lstlisting}
generic module ThreadInfoP(uint16_t stack_size, uint8_t thread_id) { 
provides {
    interface Init; // Para Inicializar as informacoes
    interface ThreadInfo; // Para exportar as Informacoes da thread
    interface ThreadFunction; // Sinaliza o evento responsavel 
                                //por executar a thread 
}}

implementation {
  uint8_t stack[stack_size];
  thread_t thread_info;

  void run_thread(void* arg) __attribute__((noinline)) {
    signal ThreadFunction.signalThreadRun(arg);
  }
  
  error_t init() {
    thread_info.next_thread = NULL;
    thread_info.id = thread_id;
    thread_info.init_block = NULL;
    thread_info.stack_ptr = (stack_ptr_t)(STACK_TOP(stack, sizeof(stack)));
    thread_info.state = TOSTHREAD_STATE_INACTIVE;
    thread_info.mutex_count = 0;
    thread_info.start_ptr = run_thread;
    thread_info.start_arg_ptr = NULL;
    thread_info.syscall = NULL;
    return SUCCESS;
  }

  ... 
}
\end{lstlisting} 

\paragraph{\textit{system/StaticThreadP.nc}:}\label{StaticThreadC}
Tem como principal objetivo servir de interface entre uma thread específica e o escalonador. Por exemplo, se
StaticThreadC recebe um comando de pausa, este é repassado para o escalonador executar. Também termina de inicializar a
thread e sinaliza o evento \textit{Thread.run}.
\begin{lstlisting}
module StaticThreadP.nc { ... }
implementation {

error_t init(uint8_t id, void* arg) {                                   
    error_t r1, r2;                                                       
    thread_t* thread_info = call ThreadInfo.get[id]();                    
    thread_info->start_arg_ptr = arg;                                     
    thread_info->mutex_count = 0;                                         
    thread_info->next_thread = NULL;                                      
    r1 = call ThreadInfo.reset[id]();                                     
    r2 = call ThreadScheduler.initThread(id);                             
    return ecombine(r1, r2);                                              
}  

event void ThreadFunction.signalThreadRun[uint8_t id](void *arg) {
    signal Thread.run[id](arg);
}

command error_t Thread.start[uint8_t id](void* arg) {
    atomic {
        if( init(id, arg) == SUCCESS ) {
            error_t e = call ThreadScheduler.startThread(id);
            if(e == SUCCESS)
                signal ThreadNotification.justCreated[id]();
            return e;
        }
    }
    return FAIL;

    ... Continuacao da implementacao da interface thread ...
    ... Todos os comandos sao simplesmente passados para o ...
    ... equivalente no ThreadScheduler ...
}

\end{lstlisting}

\paragraph{\textit{system/ThreadC.nc}:}
Esta configuração é a ``interface'' da thread com o usuário e com o escalonador. Primeiramente, é ela que prove a 
interface \textit{interfaces/Thread.nc}, portanto o programador deve codificar o tratador do evento 
\textit{Thread.run} e amarrá-lo a este componente. Em segundo lugar, conecta entre si todos os componentes 
importantes para o gerenciamento. Os principais são \textit{system/MainC} para inicialização da thread no \textit{boot} do sistema,
 \textit{system/ThreadInfoP.nc} como visto em \ref{ThreadInfoP}, e \textit{system/StaticThreadC.nc} como visto em
\ref{StaticThreadC}. A figura abaixo permite uma melhor visualização. As elipses são interfaces, os retângulos são
componentes e as setas indicam qual interface liga os dois componentes.

\includegraphics[scale=0.5]{images/tos-lib-tosthreads-system-ThreadC.png}

\paragraph{\textit{chips/atm128/chip\_thread.h}:}
Antes de expor as funções do escalonador de threads, é importante expor algumas macros de baixo nível que realizam a
troca de contexto. Para guardar o contexto de hardware da thread, criaram a estrutura \textit{thread\_regs\_t}.
\begin{lstlisting}
typedef struct thread_regs {
    uint8_t status;
    uint8_t r0;
    ...
    uint8_t r31;
} thread_regs_t;
\end{lstlisting}
Existem também algumas macros para salvar e restaurar estes registradores.
\begin{lstlisting}
 #define SAVE_STATUS(t)                              \
    __asm__("in %0,__SREG__ \n\t" : "=r" ((t)->regs.status) : );

//Save General Purpose Registers
#define SAVE_GPR(t)                                      \
    __asm__("mov %0,r0 \n\t" : "=r" ((t)->regs.r0) : );  \
    ...

//Save stack pointer
#define SAVE_STACK_PTR(t)             \
    __asm__("in %A0, __SP_L__\n\t"    \
    "in %B0, __SP_H__\n\t"            \
    :"=r"((t)->stack_ptr) : );

#define SAVE_TCB(t) \
   SAVE_GPR(t);      \
   SAVE_STATUS(t);   \
   SAVE_STACK_PTR(t) 

//Definicao das macros de restauracao
...

#define SWITCH_CONTEXTS(from, to) \
   SAVE_TCB(from);                 \
   RESTORE_TCB(to)
\end{lstlisting}
Por último, são definidas duas macros para preparação da thread.
%!!!Ainda não descobri para que serve isso exatamente!!!
%O endereço de uma função será colocado no topo da pilha da thread. Pra que ?
%O status é salvo quando SP está apontando para a pilha da thread atual e não para a pilha da que está sendo preparada.
\begin{lstlisting}
 #define SWAP_STACK_PTR(OLD, NEW) \
   __asm__("in %A0, __SP_L__\n\t in %B0, __SP_H__":"=r"(OLD):);\
   __asm__("out __SP_H__,%B0\n\t out __SP_L__,%A0"::"r"(NEW))
 
#define PREPARE_THREAD(t, thread_ptr)                      \
{  
   uint16_t temp;                                        \
   SWAP_STACK_PTR(temp, (t)->stack_ptr);                 \
   __asm__("push %A0\n push %B0"::"r"(&(thread_ptr)));   \
   SWAP_STACK_PTR((t)->stack_ptr, temp);                 \
   SAVE_STATUS(t)                                        \
}
\end{lstlisting}

\paragraph{\textit{system/TinyThreadSchedulerP.nc}:}
Durante a inicialização do sistema muitas inicializações são feitas através da interface \textit{Init} amarrada ao
compontente \textit{MainC}. Isso ocorre com a \textit{system/StaticThreadP.nc}. Como visto acima, durante a execução
desta função, o escalonador é chamado através do comando a seguir.
\begin{lstlisting}
command error_t ThreadScheduler.initThread(uint8_t id) {
    thread_t* t = (call ThreadInfo.get[id]());
    t->state = TOSTHREAD_STATE_INACTIVE;
    t->init_block = current_thread->init_block;
    call BitArrayUtils.clrArray(t->joinedOnMe, sizeof(t->joinedOnMe));
    PREPARE_THREAD(t, threadWrapper);
        //O codigo abaixo e' definicao da macro PREPARE_THREAD,
        //inserido aqui para facilitar o entendimento do codigo.
        //uint16_t temp;                                        \
        //SWAP_STACK_PTR(temp, (t)->stack_ptr);                 \
        //__asm__("push %A0\n push %B0"::"r"(&(threadWrapper)));   \
        //SWAP_STACK_PTR((t)->stack_ptr, temp);                 \
        //SAVE_STATUS(t)   
    return SUCCESS;
}
\end{lstlisting}
É importante notar que na macro \textit{PREPARE\_THREAD()}, o endereço da função \textit{threadWrapper} está sendo
empilhado na pilha da thread. Esta função encapsula a chamada para a execução da thread.
\begin{lstlisting}
void threadWrapper() __attribute__((naked, noinline)) {
    thread_t* t;
    atomic t = current_thread;

    __nesc_enable_interrupt();
    (*(t->start_ptr))(t->start_arg_ptr);

    atomic {
        stop(t);
        sleepWhileIdle();
        scheduleNextThread();
        restoreThread();
    }
}
\end{lstlisting}

No laço principal do escalonador de tarefas, quando não há mais nada para executar, a thread atual é suspensa. Com isso
o controle é passado para o escalonador de threads através do comando \textit{suspendCurrentThread()}. Na demostração de
código abaixo, algumas chamadas a funções são substituídas pelo seus corpos, para facilitar o entendimento.
\begin{lstlisting}
async command error_t ThreadScheduler.suspendCurrentThread() {
    atomic {
        if(current_thread->state == TOSTHREAD_STATE_ACTIVE) {
            current_thread->state = TOSTHREAD_STATE_SUSPENDED;
            //suspend(current_thread);
            #ifdef TOSTHREADS_TIMER_OPTIMIZATION
                num_runnable_threads--;
                post alarmTask();
            #endif
            sleepWhileIdle();
            //interrupt(current_thread);
            yielding_thread = current_thread;
            //scheduleNextThread();
            if(tos_thread->state == TOSTHREAD_STATE_READY)
                current_thread = tos_thread;
            else
                current_thread = call ThreadQueue.dequeue(&ready_queue);

            current_thread->state = TOSTHREAD_STATE_ACTIVE;
            //fim scheduleNextThread();

            if(current_thread != yielding_thread) {
                //switchThreads();
                void switchThreads() __attribute__((noinline)) {
                    SWITCH_CONTEXTS(yielding_thread, current_thread);
                 }
                //fim switchThreads();
            }
            //fim interrupt(...)
            //fim suspend(current_thread);
            return SUCCESS;
        }
        return FAIL;
    }
}
\end{lstlisting}
É muito importante notar que a função \textit{switchThreads()} não é \textit{inline}. Isso significa que os valores dos
registradores serão empilhados. Haverá então uma troca de contexto e o registrador SP apontará para a pilha da nova
thread. Por último, a função \textit{switchThreads()} retornará para o endereço que está no topo da nova pilha. Este
novo endereço, como visto acima, aponta para a função \textit{threadWrapper()}. Esta por sua vez, através de uma função
e duas sinalizações executa a thread.


\paragraph{Chamadas de sistema}
A seguir mostraremos detalhes da implementação de uma chamada de sistema. Para isso utilizaremos como exemplo a chamada
\textit{BlockingAMReceiver}, que bloqueia uma thread até o recebimento de uma mensagem, ou até o termino de tempo de
espera.

A chamada é feita utilizando o comando \textit{call BlockingReceive.receive(\&mensagemASerRecebida, timeout)}. 
A mensagem recebida será inserida no endereço de memória passado como primeiro parâmetro, e o retorno indicará se houve
sucesso ou não no recebimento da mesma.

Este comando primeiramente aloca espaço na pilha para os dados da chamada de sistema e para os
parâmetros. Como esta chamada pode ser feita com diferentes identificadores de mensagens
ativas, é preciso utilizar uma fila com as chamadas de sistema ativas.
Depois, é verificado se existe tempo máximo de espera, ou não, para chamar o comando \textit{SystemCall.start()}.
Por último, quando a chamada é completada, ela é retirada da fila, e o comando retorna.
\begin{lstlisting}
//Chamada bloqueante
command error_t BlockingReceive.receive[uint8_t am_id](message_t* m,
                                                    uint32_t timeout) {
    syscall_t s;
    params_t p;

    atomic {
        if((blockForAny == TRUE) ||
           (call SystemCallQueue.find(&am_queue, am_id) != NULL))
            return EBUSY;
        call SystemCallQueue.enqueue(&am_queue, &s);
    }

    p.msg = m;
    p.timeout = timeout;
    atomic {
        p.error = EBUSY;
        if(timeout != 0)
            call SystemCall.start(&timerTask, &s, am_id, &p);
        else
            call SystemCall.start(SYSCALL_WAIT_ON_EVENT, &s, am_id, &p);
    }

    atomic {
        call SystemCallQueue.remove(&am_queue, &s);
        return p.error;
    }   
}
\end{lstlisting}

O comando \textit{SystemCall.start} é o responsável por bloquear e armazenar as informações da thread que invocou a
chamada.
Dependendo do tipo de chamada de sistema, a thread de kernel pode acordar ou não. No caso de uma chamada que
simplesmente espera por um evento, como o recebimento de uma mensagem por rádio, o kernel não é acordado.
Porém, se a chamada precisa executar um comando, como o envio de uma mensagem, este é executado, pelo kernel, através de uma
tarefa. Portanto é preciso postar esta tarefa e acordar o kernel.
\begin{lstlisting}
command error_t SystemCall.start(void* syscall_ptr,
                                syscall_t* s, 
                                syscall_id_t id, 
                                void* p) {
    atomic {

        current_call = s; 
        current_call->id = id;
        current_call->thread = call ThreadScheduler.currentThreadInfo();
        current_call->thread->syscall = s;
        current_call->params = p;

        if(syscall_ptr != SYSCALL_WAIT_ON_EVENT) {
            current_call->syscall_ptr = syscall_ptr;
            post threadTask();
            call ThreadScheduler.wakeupThread(TOSTHREAD_TOS_THREAD_ID);
        }

        return call ThreadScheduler.suspendCurrentThread();
    }
}
\end{lstlisting}

No exemplo da chamada \textit{BlockingAMReceive.receive}, caso tenha sido determinado um \textit{timeout}, o temporizador deste
\textit{timout} será inicializado através desta tarefa.
\begin{lstlisting}
//Temporizador responsavel por calcular o timeout.
void timerTask(syscall_t* s) {
    params_t* p = s->params;
    call Timer.startOneShot[s->thread->id](*(p->timeout));  
}

//Tarefa que chama a funcao da chamada de sistema.
task void threadTask() {
    (*(current_call->syscall_ptr))(current_call);
}
\end{lstlisting}

Após o kernel ter executado a primeira fase do serviço, é preciso esperar pela segunda fase. No exemplo sendo utilizado
aqui, a segunda fase pode ser o evento correspondente ao recebimento de uma mensagem (\textit{event message\_t*
Receive.receive}), ou correspondente ao termino do tempo de espera (\textit{event void Timer.fired}).
Nos dois casos, os tratadores dos eventos são responsáveis por copiar o resultado (mensagem recebida e/ou resposta de
erro) para a variável passada como parâmetro para a chamada de sistema.
\begin{lstlisting}
event message_t* Receive.receive[uint8_t am_id](message_t* m,
                                                void* payload,
                                                uint8_t len) {
    syscall_t* s;
    params_t* p;

    if(blockForAny == TRUE)
        s = call SystemCallQueue.find(&am_queue, INVALID_ID);
    else
        s = call SystemCallQueue.find(&am_queue, am_id);
    if(s == NULL) return m;

    p = s->params;
    if( (p->error == EBUSY) ) {
        call Timer.stop[s->thread->id]();
        *(p->msg) = *m;
        p->error = SUCCESS;
        call SystemCall.finish(s);
    }
    return m;
}

event void Timer.fired[uint8_t id]() {
    thread_t* t = call ThreadScheduler.threadInfo(id);
    params_t* p = t->syscall->params;
    if( (p->error == EBUSY) ) {
        p->error = FAIL;
        call SystemCall.finish(t->syscall);
    }
}
\end{lstlisting}

Como são invocadas muitas funções, reunimos na listagem abaixo todas as passagem do ponteiro \textit{params\_t* p}, para
facilitar o entendimento.
\begin{lstlisting}
call BlockingReceive.receive(&mensagemASerRecebida, timeout);

command error_t BlockingReceive.receive[uint8_t am_id](message_t* m,
                                                    uint32_t timeout) {
    syscall_t s;
    params_t p;
    //...
    call SystemCallQueue.enqueue(&am_queue, &s);
    //...
    p.msg = m;
    //...
    call SystemCall.start(SYSCALL_WAIT_ON_EVENT, &s,am_id ,&p) ;
}

command error_t SystemCall.start(void* syscall_ptr,
                                syscall_t* s, 
                                syscall_id_t id, 
                                void* p) {
    current_call = s;
    //...
    current_call->params = p;
    //...
}

event message_t* Receive.receive[uint8_t am_id](message_t* m,
                                                void* payload,
                                                uint8_t len) {
    syscall_t* s;
    params_t* p;

    //...
    s = call SystemCallQueue.find(&am_queue, am_id);
    //...
    p = s->params;
    //...
    *(p->msg) = *m;
}
\end{lstlisting}


\subsection{Co-rotinas para o TinyOS}\label{modelo-corotinas}
O modelo que decidimos implementar foi um descrito por Ana Moura em sua tese de doutorado\cite[s. 6.2]{Moura/04}.
Neste modelo existe uma co-rotina principal que é responsável por escalonar as outras co-rotinas. 
    
    \subsubsection{Exemplo de aplicação produtor/consumidor}
    Nesta seção ilustraremos o uso de co-rotinas no TinyOS, por meio de uma aplicação que utiliza o modelo produtor/consumidor.

Ao codificar o módulo principal, é preciso definir co-rotinas serão utilizadas.
\lstinputlisting[firstline=1, lastline=7, firstnumber=1]{srcs/BenchmarkCCoro.nc}

O próximo passo é ativar as primeiras co-rotinas
\lstinputlisting[firstline=25, lastline=31, firstnumber=25]{srcs/BenchmarkCCoro.nc}

A seguir a co-rotina responsável por criar os produtos. Aqui podemos ver como é feito o uso do comando \textit{yield()},
responsável por ceder o controle.
\lstinputlisting[firstline=54, lastline=68, firstnumber=54]{srcs/BenchmarkCCoro.nc}

A co-rotina consumidora também é a responsável por ativar a co-rotina que terminará de calcular o tempo de execução de todo o
programa. É importante notar que a co-rotina \textit{SerialSender} não será executada imediatamente. Ela entrará em uma
fila, e será executada quando for escalonada. Obedecendo o modelo adodato (\ref{modelo-adotado}).
\lstinputlisting[firstline=70, lastline=85, firstnumber=70]{srcs/BenchmarkCCoro.nc}

\textit{SerialSender} é responsável por calcular o tempo final de execução, e enviar este valor pela porta serial para um computador.
Repare que as mesmas chamadas de sistema utilizadas para threads, são utilizadas aqui.
\lstinputlisting[firstline=31, lastline=51, firstnumber=31]{srcs/BenchmarkCCoro.nc}

Na configuração, é preciso declarar os componentes responsáveis pelas co-rotinas, além de definir o tamanho de suas pilhas.
\lstinputlisting[firstline=1, lastline=7]{srcs/BenchmarkAppCCoro.nc}
\lstinputlisting[firstline=21, lastline=23, firstnumber=21]{srcs/BenchmarkAppCCoro.nc}


    \subsubsection{Implementação de co-rotinas para o TinyOS}
    Nossa implementação utilizou como base a extensão \textit{TOSThreads}, vista na seção \ref{TOSThreads}. O primeiro passo
foi criar uma cópia do diretório desta extensão e um novo \textit{target} referente a este diretório para o \textit{make} do TinyOS.

Na implementação do \textit{TOSThreads} existem dois casos em que ocorre preempção: termino do \textit{timeslice} e
acontecimento de uma interrupção de hardware. Portanto foi preciso modificar esses dois casos.
A primeira alteração foi retirar o limite de tempo de execução de cada thread. Para isso o temporizador responsável por
essa contagem foi desabilitado.
A segunda alteração foi criar um novo tipo de interrupção, que chamamos de interrupção curta. Originalmente, no
\textit{TOSThreads}, quando o tratador de interrupção postava uma tarefa, o kernel assumia o controle, executava a
tarefa e escalonava a próxima thread da fila. Na nossa implementação, após o kernel executar a tarefa, a thread que foi
originalmente interrompida volta a executar. Para isso, foi criado um novo comando no escalonador de threads:
\textit{brieflyInterruptCurrentThread()}
\begin{lstlisting}
async command error_t ThreadScheduler.brieflyInterruptCurrentThread() {
    atomic {
        if(current_thread->state == TOSTHREAD_STATE_ACTIVE) {
            briefly_interrupted_thread = current_thread;
            briefly_interrupted_thread->state =
                TOSTHREAD_STATE_BRIEFLYINTERRUPTED;
            interrupt(current_thread);
            return SUCCESS;
        }
        return FAIL;
    }
}

/* schedule_next_thread()
 * This routine does the job of deciding which thread should run next.
 * Should be complete as is.  Add functionality to getNextThreadId() 
 * if you need to change the actual scheduling policy.
 */
    void scheduleNextThread() {
        if(tos_thread->state == TOSTHREAD_STATE_READY)
            current_thread = tos_thread;
        else if (briefly_interrupted_thread != NULL)
        {
            current_thread = briefly_interrupted_thread;
            briefly_interrupted_thread = NULL;
        }
        else
            current_thread = call ThreadQueue.dequeue(&ready_queue);

        current_thread->state = TOSTHREAD_STATE_ACTIVE;
    }

\end{lstlisting}

Uma vez excluída a preempção, o próximo passo foi modificar a interface da thread para permitir passagem de controle ao
escalonador. Para isso, foi criado o comando \textit{yield()}. 
\begin{lstlisting}[caption=Arquivo interfaces/Thread.nc]
interface Thread {
    ...
    command error_t yield();
    ...
}

//Arquivo: system/StaticThreadP.nc
module StaticThreadP {
    ...
    command error_t Thread.yield[uint8_t id]() { 
        return call ThreadScheduler.interruptCurrentThread(); 
    } 
    ...
}
\end{lstlisting}

%!!! Que tal adicionar um apendice mostrando o passo-a-passo da implementação de co-rotinas? !!!




\subsection{Experimentos e resultados obtidos}
Com o objetivo de comparar o desempenho da implementação de co-rotinas com a biblioteca \textit{TOSThread}, foram
desenvolvidas duas aplicações para implementar o problema do produtor-consumidor.
Uma utilizando \textit{threads} (anexo~\ref{a:appTesteThread}), e outra utilizando co-rotinas (anexo~\ref{a:appTesteCoro}).
Foram utilizados uma linha de execução para o produtor, e outra para o consumidor, e um \textit{buffer} de tamanho
único. Para simular o tempo de processamento da produção e do consumo de uma unidade, foi implementado um laço de cem iterações, onde
cada passo executa uma operação aritmética. Após consumir mil produtos, uma nova linha de execução é ativada, para
calcular o tempo de execução.
%\begin{figure}
%\centering
%\includegraphics[scale=0.4]{images/ProdutorConsumidorTinyOS.png}
%\caption{Maquina de estados - Produtor/Consumidor}
%\end{figure}

O tempo de execução foi medido em uma plataforma \textit{MicaZ}, utilizando o temporizador
\textit{Counter<TMicro,uint32\_t>}, utilizando uma precisão de microsegundos. 
Os valores medidos não variaram mais de uma unidade entre as diferentes execuções e cada cenário foi executado dez
vezes.

No primeiro experimento, variamos a quantidade de produto, referente ao laço: 
\begin{lstlisting}
for (num_prods = 0 ; num_prods < 1000; num_prods++)
\end{lstlisting}
No segundo experimentos, variamos a quantidade de operações realizadas para simular a produção e o consumo, referente ao
laço:
\begin{lstlisting}
for (j = 0 ; j < 100; j++)
    counter *= 3;
\end{lstlisting}

Na figura \ref{grafico-experimentos-corotinas} podemos ver um gráfico com os resultados.
\begin{figure}
\centering
\includegraphics[scale=0.7]{images/grafico_coroXthreads2.png}
\includegraphics[scale=0.7]{images/legenda.png}
\caption{Gráfico dos experimentos de threads e co-rotinas}
\label{grafico-experimentos-corotinas}
\end{figure}


%Para variar a carga, foram utilizadas diferentes operações aritiméticas.
%O tempo de execução foi medido em uma plataforma \textit{MicaZ}, utilizando o temporizador
%\textit{Counter<TMicro,uint32\_t>}, utilizando uma precisão de microsegundos.
%Os valores medidos não variaram mais de uma unidade entre diferentes execuções.
%\begin{center}
%    \begin{tabular}{ | l | l | l | l | l | p{5cm} |}
%    \hline
%    Operação    & x += 1 & x *= 2 & x *= 3 & x *= 5 \\ \hline
%    Threads     & 380000 & 490000 & 530000 & 563000 \\ \hline 
%    Co-rotinas  & 151000 & 252000 & 289000 & 314000 \\ \hline 
%    \end{tabular}
%\end{center}


%----------------------------------------------------------------------------------------------------
\section{Conclusões}\label{conclusoes}
As redes de sensores sem fio podem ser aplicadas em diversas áreas, por exemplo, monitoramento de oscilações 
e movimentos de pontes, observação de vulcões ativos, previsão de incêndio em florestas, entre outras. 
Muitas dessas aplicações podem atingir alta complexidade, exigindo a construção de algoritmos robustos, 
como roteamento de pacotes diferenciado.
Os escalonadores desenvolvidos neste trabalho poderão ajudar os desenvolvedores dessas aplicações complexas,
oferecendo maior flexibilidade no projeto das soluções, como 
a possibilidade de priorizar certas atividades da aplicação (comunicação via rádio ou serial, sensoriamento, etc.).
Através da análise dos experimentos realizados, cabe ao desenvolvedor decidir se a flexibilidade oferecida compensará o
\textit{overhead} gerado. Alguns dos pontos que devem ser levados em conta são: a quatidade de tarefas utilizadas, a
necessidade ou não de eliminar \textit{starvation}, e a complexidade da programação do algoritmo sem o uso dos
escalonadores propostos.

Sem um fluxo contínuo de execução, sobre a perspectiva do programador, as aplicações complexas ficam difíceis de
implementar e entender. O modelo de \textit{threads} oferecido no 
TinyOS 2.1.X\cite{TEP134} facilita este problema. Entretanto, por ser um modelo preemptivo, o custo de
gerência das threads pode implicar em queda de desempenho das aplicações. 
Com a implementação de um mecanismo de cooperação baseado em co-rotinas oferecemos uma alternativa ao programador, com
um custo menor.





%------------------------------------------------------------------------------

\section{Trabalhos Futuros}


%------------------
\section{Apêndice}
\appendix
\section{Extensão para o Simulador TOSSIM}
Para que o escalonador funcione corretamente no simulador TOSSIM é preciso adicionar funções que lidam com eventos no
simulador. Essas funções foram retiradas do arquivo
\textit{tinyOS-root-dir/tos/lib/tossim/SimSchedulerBasicP.nc}.
Primeiro é preciso alterar a implementação das interfaces de tarefa e da interface \textit{Scheduler}, criando as
funções e variáveis do código abaixo:
\begin{lstlisting}[frame=single]
  bool sim_scheduler_event_pending = FALSE;
  sim_event_t sim_scheduler_event;

  int sim_config_task_latency() {return 100;}

  void sim_scheduler_submit_event() {
    if (sim_scheduler_event_pending == FALSE) {
      sim_scheduler_event.time = sim_time() + sim_config_task_latency();
      sim_queue_insert(&sim_scheduler_event);
      sim_scheduler_event_pending = TRUE;
    }
  }

  void sim_scheduler_event_handle(sim_event_t* e) {
    sim_scheduler_event_pending = FALSE;
    if (call Scheduler.runNextTask()) 
      sim_scheduler_submit_event();
  }

  void sim_scheduler_event_init(sim_event_t* e) {
    e->mote = sim_node();
    e->force = 0;
    e->data = NULL;
    e->handle = sim_scheduler_event_handle;
    e->cleanup = sim_queue_cleanup_none;
  }
\end{lstlisting}

Depois, no comando \textit{init()} deve-se adicionar:
\begin{lstlisting}[frame=single]
  sim_scheduler_event_pending = FALSE;
  sim_scheduler_event_init(&sim_scheduler_event);
\end{lstlisting}
E, por último, nos comandos \textit{postTask()}, deve-se adicionar:
\begin{lstlisting}[frame=single]
  sim_scheduler_submit_event();
\end{lstlisting}

\label{a:escalonador-tossim}
%--------------------------------------------------------------------------------------
\section{Anexos}
\subsection{Blink}\label{a:Blink}
    \subsubsection{BlinkC.nc:}
    \lstinputlisting{srcs/BlinkC.nc}

    \subsubsection{BlinkAppC.nc:}
    \lstinputlisting{srcs/BlinkAppC.nc}

\subsection{Aplicação de Teste do Escalonador Padrão}\label{a:appTestePadrao}
    \subsubsection{aplicacaoTesteC.nc:}
    \lstinputlisting{srcs/aplicacaoTesteCPadrao.nc}

    \subsubsection{aplicacaoTesteAppC.nc:}
    \lstinputlisting{srcs/aplicacaoTesteAppC.nc}

\subsection{Aplicação de Teste do Escalonador com Prioridades}\label{a:appTeste}
    \subsubsection{aplicacaoTesteC.nc:}
    \lstinputlisting{srcs/aplicacaoTesteC.nc}

    \subsubsection{aplicacaoTesteAppC.nc:}
    \lstinputlisting{srcs/aplicacaoTesteAppC.nc}

\subsection{Aplicação de Teste do Escalonador Multi-nível}\label{a:appTesteMulti}
    \subsubsection{aplicacaoTesteC.nc:}
    \lstinputlisting{srcs/aplicacaoTesteCMultinivel.nc}

    \subsubsection{aplicacaoTesteAppC.nc:}
    \lstinputlisting{srcs/aplicacaoTesteAppCMultinivel.nc}

\subsection{Aplicação de Teste de Threads}\label{a:appTesteThread}
    \subsubsection{BenchmarkAppC.nc:}
    \lstinputlisting{srcs/BenchmarkAppCThreads.nc}

    \subsubsection{BenchmarkC.nc:}
    \lstinputlisting{srcs/BenchmarkCThreads.nc}

\subsection{Aplicação de Teste de Co-rotinas}\label{a:appTesteCoro}
    \subsubsection{BenchmarkAppC.nc:}
    \lstinputlisting{srcs/BenchmarkAppCCoro.nc}

    \subsubsection{BenchmarkC.nc:}
    \lstinputlisting{srcs/BenchmarkCCoro.nc}

\subsection{Escalonadores}
    \subsubsection{SchedulerDeadlineP.nc:}
    \lstinputlisting{srcs/SchedulerDeadlineP.nc}\label{a:edf}

    \subsubsection{SchedulerPrioridadeFilaP.nc}
    \lstinputlisting{srcs/SchedulerPrioridadeFilaP.nc}\label{a:prioridades-fila}

    \subsubsection{SchedulerPrioridadeFilaAgingP.nc}
    \lstinputlisting{srcs/SchedulerPrioridadeFilaAgingP.nc}\label{a:prioridades-fila-aging}

    \subsubsection{SchedulerPrioridadeHeapP.nc}
    \lstinputlisting{srcs/SchedulerPrioridadeHeapP.nc}\label{a:prioridades-heap}

    \subsubsection{SchedulerPrioridadeHeapAgingP.nc}
    \lstinputlisting{srcs/SchedulerPrioridadeHeapAgingP.nc}\label{a:prioridades-heap-aging}

    \subsubsection{SchedulerMultinivelP.nc}
    \lstinputlisting{srcs/SchedulerMultinivelP.nc}

%----------------------------------------------------------------------------------------------------
\bibliographystyle{unsrt}
\bibliography{ic}
\end{document}
