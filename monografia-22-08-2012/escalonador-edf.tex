Este escalonador (anexo \ref{a:edf})\footnote{O TEP 106~\cite{TEP106} disponibiliza um protótipo} aceita tarefas com deadline e 
elege aquelas com menor \textit{deadline} para executar. A interface usada para criar
esse tipo de tarefas é \textit{TaskDeadline}. O \textit{deadline} é passado por parâmetro pela função \textit{postTask}.
As tarefas básicas (\textit{TaskBasic}) também são aceitas, como recomendado pelo TEP 106\cite{TEP106}.

Em contraste, o escalonador não segue outra recomendação: não elimina a possibilidade de 
\textit{starvation} pois as tarefas
básicas só são atendidas quando não há nenhuma tarefa com \textit{deadline} esperando para executar. 
A fila é implementada da mesma forma que a do escalonador 
padrão\ref{escalonadorpadrao}, a única mudança está na inserção. Para
inserir, a fila é percorrida do começo até o fim, procurando-se o local exato de inserção.
Portanto, o custo de inserir é $\bigcirc(n)$, e o custo de retirar da fila é $\bigcirc(1)$. 
%Uma possível modificação seria utilizar uma \textit{heap}, mudando o custo de inserção  de 
%retirada para $\bigcirc(\logn)$.

%!!!explicar esse trecho melhor depois!!!
%A princípio tive problemas com o componente \textit{Counter32khzC}, 
%pois ele não existe para  a plataforma \textit{micaz}. Para poder compilar o
%escalonador foi preciso retirá-lo. Ele era usado para calcular a hora atual, e somar ao deadline. 
%Sem esse componente, temos um escalonador de prioridades (mínimo). 
