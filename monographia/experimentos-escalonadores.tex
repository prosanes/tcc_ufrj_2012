\paragraph{Experimentos com o escalonador de tarefas padrão}
Antes de começar a desenvolver outros escalonadores de tarefas, foi feito um experimento com o escalonador 
padrão que utiliza a política \textit{First in, First Out}.
Para medir a complexidade na prática, foi desenvolvida uma aplicação de teste (anexo~\ref{a:appTestePadrao}). Nela cada tarefa executa um loop de 65000
iterações, fazendo uma simples multiplicação em cada iteração. O número de tarefas variou entre 20, 50 e 100.
O tempo de execução foi medido em uma plataforma \textit{MicaZ}, utilizando o temporizador
\textit{Counter<TMicro,uint32\_t>}, utilizando uma precisão de microsegundos.
Os valores medidos não variaram mais de uma unidade entre diferentes execuções, e cada cenário foi executado dez
vezes.
\begin{center}
    \begin{tabular}{ | l | l | l | l | p{5cm} |}
    \hline
    Escalonador              & 20 Tarefas & 50 Tarefas & 100 Tarefas \\ \hline
    Escalonador Padrão       & 1366 & 1849 & 2652 \\ \hline
    \end{tabular}
\end{center}

%---------------------------------
\paragraph{Experimentos com o escalonador com prioridades}
Para avaliar o desempenho com o escalonador com prioridades foi desenvolvida a mesma aplicação de teste (anexo
\ref{a:appTeste}),
onde cada tarefa executa um loop de 65000 iterações, fazendo uma simples multiplicação em cada iteração. 
A prioridade de todas as tarefas, exceto uma, era igual, de forma que toda inserção deveria percorrer toda a fila.
A tarefa responsável por calcular o tempo de execução do experimento tinha a menor prioridade, para que esta fosse a
última a executar.
O número de tarefas variou entre 20, 50 e 100.
O tempo de execução foi medido em uma plataforma \textit{MicaZ}, utilizando o temporizador 
\textit{Counter<TMicro,uint32\_t>}, utilizando uma precisão de microsegundos. 
Os valores medidos não variaram mais de uma unidade entre diferentes execuções, e cada cenário foi executado dez
vezes.
\begin{center}
    \begin{tabular}{ | l | l | l | l | p{5cm} |}
    \hline
    Escalonador              & 20 Tarefas & 50 Tarefas & 100 Tarefas \\ \hline
    Escalonador Padrão       & 1366 & 1849 & 2652 \\ \hline 
    Fila, sem envelhecimento & 1733 & 4660 & 13721 \\ \hline 
    Heap, sem envelhecimento & 2603 & 4308 & 7486 \\ \hline
    Fila, com envelhecimento & 2278 & 7887 & 26066 \\ \hline
    Heap, com envelhecimento & 2665 & 4510 & 7887 \\ \hline
    \end{tabular}
\end{center}

Podemos perceber que, para um número pequeno de tarefas, a fila é mais eficiente que a heap.
%EXPLICAR
não é compensado o \textit{overhead} do algoritmo da heap.

